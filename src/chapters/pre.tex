In this chapter I explain each step taken during the preprocessing phase, by illustrating in details each step taken and tool used.
\section{Data Collection}
Data collection is the process of gathering information in variables of interest for answering relevant questions. 
Variables selected are the physical and chemical factors that are most associated with the formation of primary and secondary pollutant. 
Therefore, the variables are categorized in 4 different labels:
\begin{itemize}
\item Weather: These elements, such as wind speed and direction, precipitation and air temperature, changes in the epochs and can influence air pollution;
\item Pollutant: These variables represent primary and secondary pollutant related to the greenhouse effect;
\item Soil and Vegetation: Since soil and vegetation degradation are global concerns and can inluence the air propagation in the environment, data related to local morphology are collected;
\item GIS (static layers): This time-invariant layers are considered to be changeless in the time range considered. Differently from the other types which need a constant monitoring, these variables are update yearly with a lower frequency than the others;
\end{itemize}
Data chosen are open source and regularly available.
In this phase data have been collected (not by me but by other collegues of D-DUST project at this \href{https://docs.google.com/spreadsheets/d/1-5pwMSc1QlFyC8iIaA-l1fWhWtpqVio2/edit#gid=91313358}{link}) in grids from different sources and provider in Geopackages. 
\subsection{Source types}
In order to better distinguish the data sources characteristics, variables selected are labeled with 4 different types:
\begin{itemize}

\item Ground Sensor: Each ground monitoring stations belongs mainly to ARPA and ESA provides meterological and air quality data;
\item Model: data are estimated through a model built using satellite and meteorological and air quality data as input, such as european data provided by CAMS (Copernicus Atmosphere Monitoring Service);
\item Map layer: this data are time-invariant and are related to Lombardy morphology such as density of roads, population or land use; 
\item Satellite Sensor: They provide data from air quality observation mainly. Satellites provider are Sentinel-5P Tropomi and Terra \& Aqua MODIS;
\end{itemize}
\subsection{Spatial resolution}
Vector grids that are used in the D-DUST project are three and they are generated by the spatial resolution of the source provider. 

\begin{itemize}
\item Grids with 0.1° resolution with Copernicus CAMS (European);
\item Grids with 0.066° resolution based on S5P (this resolution is not included in the case of study);
\item Grids with 0.01° Grid defined with maximum one ARPA station for each cell;
\end{itemize}

Data are scaled and fit in each spatial resolution grid in order to better analyze the final output model by considering each of them.

In the next lines each variable is provided in tables, by showing its type, name and description:

\subsubsection{Meteo}
\begin{center}
\setlength{\arrayrulewidth}{1.5pt}

\begin{longtable}{ |p{2cm}|p{1.5cm}|p{2.3cm}|p{4cm}|p{1cm}|p{2cm}| } 
\hline
\textbf{Physical variable} & \textbf{Source type}  & \textbf{Variable name}  & \textbf{Description}  & \textbf{Unit}  & \textbf{Source}\\ 
\hline
\multirow{3}{4em}{Temperature} & Model  & \underline{temp\_2m} & Mean air temperature at 2 m above the land surface & K & ERA5-Land hourly data.\\ 
& Ground \newline Sensor  & \underline{temp\_lcs} &  Mean air temperature ground measurement - Low Cost Sensor ESA monitoring stations. & °C & ESA Air Quality Platform.\\ 
& Ground \newline Sensor  & \underline{temp\_st} &  Mean temperature - ARPA monitoring stations. & °C & ARPA \newline Lombardia.\\ \hline

\multirow{4}{4em}{Wind} & Model  & \underline{e\_wind} & Mean eastward wind component 10 m above the land surface & m/s & ERA5-Land hourly data.\\ 
& Ground \newline Sensor  & \underline{wind\_dir\_st} &  Wind direction from ground sensor divided in 8 sectors. These are classified into 8 categories as specified in "Notes" column. & cat & ARPA \newline Lombardia.\\ 
& Ground \newline Sensor  & \underline{n\_wind} &  Mean northward wind component 10 m above the land surface. & m/s & ERA5-Land hourly data.\\
& Ground \newline Sensor  & \underline{wind\_speed\_st} &  Mean wind speed on ground  - ARPA monitoring stations. & m/s& ARPA \newline Lombardia.\\ \hline

\multirow{2}{4em}{Precipitation} & Model  & \underline{prec} & Mean accumulated liquid and frozen water, including rain and snow, that falls to the Earth's surface. It is the sum of large-scale precipitation. & m & ERA5-Land hourly data.\\ 
& Ground \newline Sensor  & \underline{prec\_st} &  Mean precipitation in each cell in the time range - ARPA monitor stations. & mm & ARPA \newline Lombardia.\\ \hline

\multirow{2}{4em}{Air Humidity} & Ground \newline Sensor  & \underline{air\_hum\_st} & Mean air moisture measurement in the time range - ARPA monitoring stations & \% & ARPA \newline Lombardia.\\ 
& Ground \newline Sensor  & \underline{air\_hum\_lcs} &  Mean air moisture ground measurement - Low Cost Sensor ESA monitoring stations. & \% & ESA Air Quality Platform.\\ \hline

\multirow{1}{4em}{Air Pressure} & Model   & \underline{press} & Mean weight of all the air in a column vertically above the area of the Earth's surface represented at a fixed point. & Pa & ERA5-Land hourly data.\\ \hline

\multirow{1}{4em}{Solar Radiation} & Ground \newline Sensor  & \underline{press} & Global radiation measurement - ARPA monitoring station. & W/m\textsuperscript{2} & ARPA \newline Lombardia.\\ \hline

\hline
\caption{Table of Methorogical variables.}

\end{longtable}
\end{center}

\subsubsection{Pollutants}


\begin{center}
\setlength{\arrayrulewidth}{1.5pt}
\begin{longtable}{ |p{2cm}|p{1.5cm}|p{2.3cm}|p{4cm}|p{1cm}|p{2cm}| } 
\hline
\textbf{Physical variable} & \textbf{Source type}  & \textbf{Variable name}  & \textbf{Description}  & \textbf{Unit}  & \textbf{Source}\\ 
\hline
\multirow{1}{4em}{Dust} & Model  & \underline{dust} & Mean dust concentration at 0m level provided by CAMS (Ensemble Median - Analysis). & ug/m\textsuperscript{3} & CAMS Model.\\ \hline

\multirow{3}{4em}{AOD} & Satellite \newline Sensor  & \underline{aod\_055} & Mean Aerosol Optical Depth at 550nm. & - & MODIS Terra+Aqua.\\ 
& Satellite \newline Sensor  & \underline{aod\_047} &  Mean Aerosol Optical Depth at 470nm. & - & MODIS Terra+Aqua.\\ 
& Satellite \newline Sensor & \underline{uvai} &  Mean UV Aerosol Index. A positive index highlights the presence of UV absorbing aerosol (such as smoke/dust).  & - & Sentinel-5P\\ \hline

\multirow{3}{4em}{PM10} & Model  & \underline{pm10\_cams} & Mean PM10 concentration at 0m level provided by CAMS  (Ensemble Median - Analysis). & ug/m\textsuperscript{3} & CAMS Model.\\ 
& Ground \newline Sensor  & \underline{pm10\_lcs} &  Mean PM10 concentration ground measurement - Low Cost Sensor ESA monitoring stations. & ? & ESA Air Quality Platform.\\ 
& Ground \newline Sensor & \underline{pm10\_st} &  Mean PM10 concentration ground measurement - ARPA monitoring stations.  & ug/m\textsuperscript{3} & ARPA \newline Lombardia\\ \hline

\multirow{3}{4em}{PM2.5} & Model  & \underline{pm25\_cams} & Mean PM2.5 concentration at 0m level provided by CAMS  (Ensemble Median - Analysis). & ug/m\textsuperscript{3} & CAMS Model.\\ 
& Ground \newline Sensor  & \underline{pm25\_lcs} &  Mean PM2.5 concentration ground measurement - Low Cost Sensor ESA monitoring stations. & ug/m\textsuperscript{3} & ESA Air Quality Platform.\\ 
& Ground \newline Sensor & \underline{pm25\_st} &  Mean PM2.5 concentration ground measurement - ARPA monitoring stations.  & ug/m\textsuperscript{3} & ARPA \newline Lombardia\\ \hline

\multirow{3}{4em}{SO\textsubscript{2}} & Model  & \underline{so2\_cams} & Mean SO\textsubscript{2} concentration at 0m level provided by CAMS  (Ensemble Median - Analysis). & ug/m\textsuperscript{3} & CAMS Model.\\ 
& Satellite \newline Sensor  & \underline{so2\_s5p} &  Mean SO2  vertical column density at ground level. & mol/m\textsuperscript{2} & Sentinel-5P.\\ 
& Ground \newline Sensor & \underline{so2\_st} &  Mean SO\textsubscript{2} concentration ground measurement - ARPA monitoring stations.  & ug/m\textsuperscript{3} & ARPA \newline Lombardia.\\ \hline


\multirow{4}{4em}{NO\textsubscript{2}} & Model  & \underline{no2\_cams} & Mean NO\textsubscript{2} concentration at 0m level provided by CAMS  (Ensemble Median - Analysis). & ug/m\textsuperscript{3} & CAMS Model.\\ 
& Satellite \newline Sensor  & \underline{no2\_s5p} &  Mean NO2  vertical column density at ground level. & mol/m\textsuperscript{2} & Sentinel-5P.\\ 
& Ground \newline Sensor & \underline{no2\_st} &  Mean NO\textsubscript{2} concentration ground measurement - ARPA monitoring stations.  & ug/m\textsuperscript{3} & ARPA \newline Lombardia.\\ 
& Ground \newline Sensor & \underline{no2\_lcs} &  Mean NO\textsubscript{2} concentration ground measurement - Low Cost Sensor ESA monitoring stations.  & ug/m\textsuperscript{3} & ESA Air Quality Platform.\\ \hline

\multirow{1}{4em}{NO} & Model  & \underline{no2\_cams} & Mean NO concentration at 0m level provided by CAMS  (Ensemble Median - Analysis). & ug/m\textsuperscript{3} & CAMS Model.\\  \hline

\multirow{1}{4em}{NO\textsubscript{x}} & Ground \newline Sensor & \underline{nox\_st} &  Mean NO\textsubscript{x} (field: "Ossidi di Azoto") concentration ground measurement - ARPA monitoring stations  & ug/m\textsuperscript{3} & ARPA \newline Lombardia.\\ \hline

\multirow{1}{4em}{CO\textsubscript{2}} & Ground \newline Sensor & \underline{co2\_lcs} &  Mean CO2 concentration ground measurement - Low Cost Sensor ESA monitoring stations & ? & ESA Air Quality Platform.\\ \hline

\multirow{4}{4em}{CO} & Model  & \underline{co\_cams} & Mean CO concentration at 0m level provided by CAMS  (Ensemble Median - Analysis). & ug/m\textsuperscript{3} & CAMS Model.\\ 
& Satellite \newline Sensor  & \underline{co\_s5p} &  Mean CO vertically integrated column density. & mol/m\textsuperscript{2} & Sentinel-5P.\\ 
& Ground \newline Sensor & \underline{co\_st} &  Mean CO concentration ground measurement - ARPA monitoring stations.  & ug/m\textsuperscript{3} & ARPA \newline Lombardia.\\ 
& Ground \newline Sensor & \underline{co\_lcs} &  Mean CO concentration ground measurement - Low Cost Sensor ESA monitoring stations.  & ug/m\textsuperscript{3} & ESA Air Quality Platform.\\ \hline

\multirow{3}{4em}{O\textsubscript{3}} & Model  & \underline{o3\_cams} & Mean O\textsubscript{3} concentration at 0m level provided by CAMS  (Ensemble Median - Analysis). & ug/m\textsuperscript{3} & CAMS Model.\\ 
& Satellite \newline Sensor  & \underline{03\_s5p} &  Mean O\textsubscript{3} total atmospheric column  & mol/m\textsuperscript{2} & Sentinel-5P.\\ 
& Ground \newline Sensor & \underline{03\_st} &  Mean O\textsubscript{3} concentration ground measurement - ARPA monitoring stations.  & ug/m\textsuperscript{3} & ARPA \newline Lombardia.\\ 
 \hline
 
 \multirow{1}{4em}{CH\textsubscript{2}O}& Satellite \newline Sensor  & \underline{ch20\_s5p} &  Mean Formaldehyde tropospheric column number density & mol/m\textsuperscript{2} & Sentinel-5P.\\ \hline
 
\multirow{1}{4em}{NMOVOCs}& Model  & \underline{nmvocs\_cams} & Mean Non-Methane VOCs concentrations at 0m level provided by CAMS. & ug/m\textsuperscript{3} & CAMS Model.\\ \hline

\multirow{3}{4em}{NH\textsubscript{3}} & Model  & \underline{nh3\_cams} & Mean NH\textsubscript{3} concentration at 0m level provided by CAMS  (Ensemble Median - Analysis). & ug/m\textsuperscript{3} & CAMS Model.\\ 
& Satellite \newline Sensor  & \underline{nh3\_lcs} &  Mean NH\textsubscript{3} concentration ground measurement - Low Cost Sensor ESA monitoring stations  & ? & ESA Air Quality Platform.\\ 
& Ground \newline Sensor & \underline{nh3\_st} &  Mean NH\textsubscript{3} concentration ground measurement - ARPA monitoring stations.  & ug/m\textsuperscript{3} & ARPA \newline Lombardia.\\ \hline
\caption{Table of Pollutant variables.}
\end{longtable}
\end{center}

\subsubsection{Soil and Vegetation}

\begin{center}
\setlength{\arrayrulewidth}{1.5pt}
\begin{longtable}{ |p{2cm}|p{1.5cm}|p{2.3cm}|p{4cm}|p{1cm}|p{2cm}| } 
\hline
\textbf{Physical variable} & \textbf{Source type}  & \textbf{Variable name}  & \textbf{Description}  & \textbf{Unit}  & \textbf{Source}\\ 
\hline

\multirow{3}{4em}{Vegetation} & Satellite \newline Sensor  & \underline{siarlX} & Fraction of area in each cell for each agricultural use provided by SIARL Catalog for Lombardy Region. & \% & SIARL Lombardia 2019.\\ 
& Satellite \newline Sensor  & \underline{ndvi} &  Mean NDVI cell value over 16 days period & - & USGS Earth Data.\\ 
& Satellite \newline Sensor  & \underline{siarl} &  Majority class for agricultural use provided by SIARL Catalog for Lombardy Region. & cat & SIARL Lombardia 2019.\\
\hline

\multirow{5}{4em}{Soil} & Model  & \underline{soil\_moist} & Mean volume of water in soil layer 1 (0 - 7 cm) of the ECMWF Integrated Forecasting System. The surface is at 0 cm. The volumetric soil water is associated with the soil texture (or classification), soil depth, and the underlying groundwater level. & m\textsuperscript{3}/m\textsuperscript{3} & ERA5 Land Hourly Data.\\ 
& Map Layer  & \underline{soilX} &  Fraction of area for each cell containg the soil type obtained from OpenLandMap soil texture classification. & \% & OpenLandMap Soil Texture Class (USDA System).\\ 
& Map Layer  & \underline{soil\_textX} &  Mean NDVI cell value over 16 days period & \% & Basi informative dei suoli - Geoportale Lombardia.\\ 
& Map Layer  & \underline{soil} &  Majority soil type for each pixel from OpenLandMap soil texture classification . & cat & OpenLandMap Soil Texture Class (USDA System) .\\ 
& Map Layer  & \underline{soil\_text} &  Majority soil type for each pixel from Carta pedologica 250K (Lombardy Region). & cat & Basi informative dei suoli - Geoportale Lombardia.\\ 

\hline
\caption{Table of variables referred to Vegatation and Soil.}

\end{longtable}
\end{center}

\subsubsection{GIS (static layers)}

\begin{center}
\setlength{\arrayrulewidth}{1.5pt}
\begin{longtable}{ |p{2cm}|p{1.5cm}|p{2.3cm}|p{4cm}|p{1.2cm}|p{2cm}| } 
\hline
\textbf{Physical variable} & \textbf{Source type}  & \textbf{Variable name}  & \textbf{Description}  & \textbf{Unit}  & \textbf{Source}\\ 
\hline
\multirow{1}{4em}{Geometry} & Map Layer  & \underline{area} & Area of Lombardy Region vector layer in each cell. & km\textsuperscript{2} & SIARL Lombardia 2019.\\ 
& Satellite \newline Sensor  & \underline{ndvi} &  Mean NDVI cell value over 16 days period & - & - \\ \hline

\multirow{1}{4em}{Population} & Map Layer  & \underline{pop} & Population for each cell. & n° of inhabitants& Gridded Population of the World (GPW).\\ \hline

\multirow{2}{4em}{Land use and cover} & Map Layer  & \underline{dsfX} & Land use fraction for each cell containing the classification the classification provided by DUSAF Catalog (Lombardy Region). & \% (fraction for each cell) & DUSAF Lombardia 2018.\\ 
Map Layer  & \underline{dusaf} & Cover & Land Use majority class for each cell provided by DUSAF Catalog (Lombardy Region). & cat  & DUSAF Lombardia 2018.\\
\hline

\multirow{3}{4em}{Terrain} & Map Layer  & \underline{h\_mean} & DTM average elevation for each pixel. & m & Geoportale Lombardia 2019.\\ 
& Map Layer  & \underline{aspect\_major} & Aspect derived from DTM. Majority pixel aspect. & Degree North & Geoportale Lombardia 2019.\\ 
& Map Layer  & \underline{slope\_mean} & Average slope derived from DTM. & Degree North & Geoportale Lombardia 2019.\\ 

\hline

\multirow{6}{4em}{Road Infrastructures} & Map Layer  & \underline{int\_prim} & Density of intersection nodes between primary roads for each cell (including highways). & int\textsubscript{s}/km\textsuperscript{2} & Geoportale Lombardia 2019.\\ 
& Map Layer  & \underline{int\_prim\_sec} & Density of intersection nodes between primary and secondary roads for each cell. & int\textsubscript{s}/km\textsuperscript{2} & Geoportale Lombardia 2019.\\ 
& Map Layer  & \underline{int\_sec} & Density of intersection nodes between secondary roads for each cell. & int\textsubscript{s}/km\textsuperscript{2} & Geoportale Lombardia 2019.\\ 
& Map Layer  & \underline{prim\_road} & Density of primary importance roads for Lombardy Region inside for each. & km/km\textsuperscript{2} & Geoportale Lombardia 2019.\\ 
& Map Layer  & \underline{sec\_road} & Density of secondary importance roads for Lombardy Region foreach cell. & km/km\textsuperscript{2} & Geoportale Lombardia 2019.\\ 
& Map Layer  & \underline{highway} & Density of highways for Lombardy Region inside for cell divided. & km/km\textsuperscript{2} & Geoportale Lombardia 2019.\\ 
\hline

\multirow{1}{4em}{Farms} & Map Layer  & \underline{farms} & Fration of area covered by farms inside the cell. Obtained from DUSAF dataset. & \% (fraction for each cell) & DUSAF Lombardia 2018.\\ \hline
\multirow{1}{4em}{Air quality zones} & Map Layer  & \underline{aq\_zone} & Majority class of a given air quality zone in each cell. & cat  & Geoportale Lombardia.\\ \hline
\multirow{1}{4em}{Climate zones} & Map Layer  & \underline{clim\_zone} & Majority class of a given air quality zone in each cell. & cat  & - \\ \hline


\hline
\caption{Table of Static GIS variables.}

\end{longtable}
\end{center}
\subsubsection{Categorical Variable}
Categorical data are identified with names or labels given to them as value. Even if are rapresented by numbers, they don't have the same mathematical meaning as a numerical value. This type of data is discarded during the preprocessing phase, since feature selection is done exclusively on numerical input and output values. 
In the following table is explained the semantic of the values assumed.
\begin{center} 
\setlength{\arrayrulewidth}{1.5pt}
\begin{longtable}{ |p{2.5cm}|p{10cm}| } 
\hline
\textbf{Variable name} & \textbf{Note}\\ 
\hline

 \multicolumn{2}{|c|}{\textbf{Meteo}} \\
\hline
 \underline{wind\_dir\_st}  & 1 = North: 0° - 22.5° / 337.5° - 360°, \newline2 = North-East: 22.5° - 67.5°, \newline3 = East: 67.5° - 112.5°,  \newline4 = South-East: 112.5° - 157.5°, \newline5 = South: 157.5° - 202.5°, \newline6 = South-West: 202.5° - 247.5°, \newline7 = West: 247.5° - 292.5°, 8 = North-West: 292.5° - 337.5°\\ \hline
 \multicolumn{2}{|c|}{\textbf{Soil and Vegetation}} \\ \hline
 \underline{siarl} & 2 = Cereal \newline9 = Mais \newline12 = Rice\\  \hline
 \underline{soil} &  2=Cereal\newline9=Mais\newline12=Rice\\ \hline 
\underline{soil\_text} &  1 = Clay\newline2 = Silty Clay \newline3 = Sandy Clay\newline4 = Clay Loeam \newline5 = Silty Clay Loam\newline6 = Sandy Clay Loam \newline7 = Loam \newline8 = Silt Loam \newline9 = Sandy Loam \newline10 = Silt \newline 11 = Loamy Sand \newline12 = Sand.\\ \hline
 \multicolumn{2}{|c|}{\textbf{GIS (Static layers)}} \\ \hline
 \underline{dusaf} & 2 = Agricultural areas \newline3 = Wooded territories and semi-natural environments \newline4 = Wetlands \newline5 = Water bodies \newline11 = Urbanised areas \newline12 = Production facilities, large plants and communication networks \newline13 = Mining areas, landfills, construction sites, waste and abandoned land \newline14 = Non-agricultural green areas \\
\hline
 \underline{aq\_zone} & 1 = Highly urbanized plains \newline 2 = Plains \newline 3 = Prealpi, Appennino and mountains \newline 4 = Valley floor Agg. 
\newline5 = Urban agglomarated area (Milano, Bergamo, Brescia).\\
\hline
 \underline{clim\_zone} & 1= Alpi\newline 2 = Prealpi Occidentali \newline 3 = Prealpi Orientali\newline 4 = Pianura Occidentale\newline 5 =  Pianura Centrale\newline 6 = Pianura Orientale. 
 \\
\hline
\caption{Table of categorical variables with their values legend.}



\end{longtable}
\end{center}

\section{Data Cleaning}
Data has to be prepared in accordance with the supervised feature selection.
Data cleaning aims to fix problems or errors in messy data. There are many reasons data may have incorrect values, such as being corrupted, duplicated or invalid. 
This could be done by removing a rows or columns. Alternately, it might involve replacing observations with new values. 
Firstly data covariates are splitted between input (X) and output variable (Y). X represents all of variables collected in the previous part, excepting for the pollutant to be analyzed and modelled (such as PM25 or Ammonia) which is assigned to the Y variable.

In this section are underlined the main steps performed in the Data Cleaning phase.
\subsection{Nan Values}
In my work I consider as target variable PM25 and Ammonia coming from ground sensor measurament.
Air quality monitoring is usually carried out through ground sensors networks, which represents the primary air quality data source by governance. 
In the grids processed there's the problem that a given value provided by measurament tools (such as ground and satellite sensor) could be NaN. 
It's feasible since:
\begin{itemize}
\item A sensor could have no measurament for a given time epoch;
\item The set of sensor, because of its limited supply, cannot cover each cell of a grid;
\end{itemize}
In our case variable provided by ARPA and ESA ground sensor (with the label that ends with '\_st' and '\_lcs' respectevively) has many NaN cells.

However, no country in the world has yet established a monitoring network with a full satisfying coverage\cite{liu2018improve}. Even in the United States (US), which is characterized by a relatively developed PM2.5 ground monitoring network with 2500 stations has many areas unmonitored\cite{liu2018improve}. 

In order to mitigate this, I present this solution in sequence:
\begin{itemize}
\item Drop of samples having target variable with NaN value;
\item Drop of columns (values assumed by each covariates) having at least a NaN value;
\end{itemize}
Due to the fact that it results a dataset with a very limited number of sample \cite{zhang2018strategy}, I perform additionally a k-nearest neighbor classifier\cite{taunk2019brief} for adding a buffer of values close to the location of the ground stations measurament. Values added are computed using a Radial Basis function interpolation\cite{wright2003radial}.
In this way the size of the final sample, as the performance of the feature selection would increase.

\subsection{Remove of variables with low variance}
An approach for removing columns is to consider the variance of each column variable. The variance is a statistic representing the expected value of the squared deviation from the mean of a given variable X $\mu$. 
\begin{equation}
  Var(X) = E[(X-\mu)^2]
\end{equation}
The variance can be used as a filter for identifying columns to be removed from a given dataset. 
Using a feature with low-variance only adds complexity and noisy to the feature selection and the predictive.
In order to do that, I perfomed VarianceThreshold method from the scikit-learn library. In this way, features under a certain variance threshold value should be meaningless and consequentely discarded by its dataset. 
\section{Data Transformation}
Having input variables with different units (e.g. ug/m\textsuperscript{3}, °C, hours or mol/m\textsuperscript{2}) implies data at different scales. This could raise the difficulty of the problem being modeled. 
Hence, a common scale through Normalization or Standardization is needed in order to improve the data quality.
Many ML and regression algorithms perform better when numerical input and output variables are scaled to a common standard range. 
For instance, it's proved that neural networks trained with scaled data performs better in terms of MSE \cite{shanker1996effect}.
In this step, two type of transformation have been done:
\begin{itemize}
\item Normalization
\item Standardization: it consists in rescaling data following a gaussian distribution of values with mean equals to 0 and and standard deviation equals to 1. In this way 
\end{itemize}

\section{Feature Selection}




