The arrival of big data from heterogenous sources has lead to enormous impact in pollution analysis.
At the same time, techniques such as Machine Learning models become more and more the main approaches to make prediction and to detect the main factors affecting a given target variable.
In particular with data set with large sample and covariates get AI models more complex to be interpreted and explained.
One machine learning issue is the fact that usually models are a kind of “black box”: systems that evaluate output not fully understandable by humans. 
It should be important to interpret prediction, especially for critical decision such as health state diagnostics.
A way to interpret models for better reliability is to compare feature importance: giving an hierarchy of features could be very useful to distinct the most influential variables from the ones which could confound model decisions.
The challenge of this thesis is to propose an approach that carry on the concept of finding the best potential predictive variables based on the weighted score.
The advantage of using it's that the feature selected reduce redundancy and increase interpretability. 
Subsequent predictive models should have been improved in terms of accuracy and confidence of the results.
In detail, this thesis present a set of tools for preprocessing data and selecting variables with filter, wrapper and embedded methods.
Some ML models are also developed with the aim to compare contribution for the accuracy of the results.
Experimental results obtained by the tests with different configurations and resolutions are analysed and compared by analogous scientific literature.
This thesis wants to contribute to the preprocessing phase which is performed for Machine Learning models, considering it essential before regression model training.
