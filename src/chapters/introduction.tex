Science demonstrated that deterioration of ambient air quality, due to the growing concentration of pollutant in the atmosphere, has cause a significant increment of deaths in the world.  
Pollutants such as particulate matter, carbon monoxide, ozone, nitrogen dioxide and sulfur dioxide, cause respiratory diseases, and are important sources of mortality. 
Almost the entire global population (99\%) breathes air that exceeds WHO air quality limits, and threatens their health.\par
In Europe air is getting cleaner, but persistent pollution, especially in cities, is damaging people’s health. The latest reports are based on the European Environment Agency’s (EEA), which shows that exposure to air pollution caused around 400,000 early deaths in the European Union (EU) in 2016.\par
One of the most harmful pollutants is the \textbf{particulate matter (PM)} which, according to is diamater (2.5\textmu m or 10 \textmu m), can get deep into your lungs or even get into your bloodstream.\par
Most of the particles come from chemical reactions such as sulfur dioxide and nitrogen oxides, which are pollutants emitted from power plants, industries and automobiles.\par
However, a significant sources of PM are the chemical reactions generated by intensive farming.
In particular this is a significant issue in the Po Valley, where intensive agricultural activity is very employed.


\begin{comment}
 
In view of the above, the D-DUST project focuses on the development of new means to improve both understanding and local monitoring of farming-related PM. The project will primarily consider the Po Valley portion belonging to the Lombardy Region as a testbed for the activities. The aim of D-DUST is to assess the contribution (in terms operability, cost-effectiveness, and accuracy improvement) deriving by the systematic integration of non-conventional data, with a focus on satellite-based PM estimates, into traditional PM monitoring frameworks based on fixed ground-sensors. Data ingestion to support traditional PM monitoring and modelling will take the best advantage of data science techniques throughout the combination of machine-learning and geostatistical models. The satellite-based PM prediction modelling aims at improving the granularity of the available ground-based PM measurements to enable exposure and source apportionment models to better account for time and space variability of exposure patterns. Periodical high- detailed PM observations from on-site sampling with chemical characterization will support the modelling activities by providing valuable information for models’ training and validation and targeting intensive farming activities as the focal emission source. Reproducibility of the research activities will be promoted by openly distributing most meaningful analysis data and leveraging the use of free and open-source software technologies along the whole data analysis process.
The ultimate project goal is to develop data-driven best-practices to be transferred to operational air quality monitoring and policymaking, by spelling out data requirements, analysis patterns, software and hardware equipment, and technical skills that are demanded to operate the proposed methods. The availability of these new informational assets
 in order to evaluate a model capable of making predictions through machine learning tecniques. \par
Indeed, gound-based measurement stations cannot cover each areas. So in order to provide estimates and forecasts that could be replicated satellite platforms data are collected.

\end{comment}
\section{D-Dust}
In this context, the \textbf{D-DUST project} (Data-driven moDelling of particUlate with Satellite Technology aid) aims to provide knowledge about the impact of agricultural and livestock activities on pollutants in the Po Valley (Northern Italy).\par 
For doing that, data from ground sensor are combined with contribution provided by satellite platforms and, with the use of data science tecniques like machine-learning and geostatistical models, provide meaningfull information related to the contribution of intensive farming on pollution.
The final target of the project is to provide a data-driven best-practices to policymakers, farming operators and citizens in order to minimize the production processes' effects on air quality.


\section{Overview}


