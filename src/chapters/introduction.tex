Science demonstrated that deterioration of ambient air quality, due to the growing concentration of pollutant in the atmosphere, has caused a significant increment of deaths in the world.\par  
Pollutants such as particulate matter, ozone, carbon monoxide and ammonia cause respiratory diseases, and are important sources of mortality.
Almost the entire global population (99\%) breathes air that exceeds WHO air quality limits, and threatens their health.\newline
In Europe air is getting cleaner, but persistent pollution, especially in cities, is damaging people’s health. One of the last reports which is based on the European Environment Agency’s (EEA), shows that exposure to air pollution caused around 500,000 early deaths in the European Union (EU) in 2018 \cite{european2018air}.\par
One of the most harmful pollutants is the \textbf{particulate matter (PM25 or PM10)} which can get deep into your lungs or even get into your bloodstream.\newline
Most of the particles come from chemical reactions such as sulphur dioxide and nitrogen oxides, which are pollutants emitted from power plants, industries and automobiles.\par
However, a significant sources of PM are the chemical reactions generated by intensive farming \cite{burkart2007diffuse}.
In particular this is a relevant issue in the Po Valley, where intensive agricultural activity is very employed.\par
In this context, human civilization is trying to limit pollution and improve environment with use of technology.\newline
Technology is helping to clean up air pollution, with data analytics-based solutions helping to make our cities healthier places to live.\newline
Monitoring, analysing and predicting the air quality in urban areas is one of the effective solutions for coping with the climate change problem.\par
The advent of modern Artificial Intelligence (AI) techniques such as Machine Learning (ML) can be considered as new possibilities for researchers to find solutions to various problems affecting air quality and climate change.
\bigskip
In this context, the \textbf{D-DUST project} (Data-driven moDelling of particUlate with Satellite Technology aid), funded by Fondazione Cariplo’s ‘Data Science for Science and Society’ call for proposals, counts on Politecnico di Milano, Department of Civil and Environmental Engineering (DICA) as lead partner.\newline
D-DUST aims to provide knowledge about the impact of agricultural and livestock activities on pollutants in the Po Valley (Northern Italy).\par 
For reaching the goal, data from ground sensor are combined with contribution provided by satellite platforms and, with the use of data science techniques like machine-learning and geostatistical models, provide meaningful information related to the contribution of intensive farming on pollution.\par
The last target of the project is to provide a data-driven best-practices to policymakers, farming operators and citizens in order to minimize the production processes' effects on air quality.
\bigskip
In this thesis we propose an ensemble approach for analysing data and provide useful information regarding intense agricultural activity through selection of the most remarkable covariates that impact on PM25 and NH3 pollutants. 
The final step is to build a model skilled to estimate pollutant estimation locally, better than global scale model.  