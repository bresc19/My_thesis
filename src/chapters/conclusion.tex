Based on the results achieved in the previous chapter, we can conclude that there are Machine Learning models that perform better than the others.\\
Another important issue is the importance of the number of observations used from training data. In order to have enhanced estimation, the increment of these is crucial.
Nevertheless, this case of study was not brought to build a proper model for air pollution forecast, but how the selection of relevant features affects the model performance. 
The present study confirmed the findings about the model performance which increases if a feature selection is applied.
Using feature selection, it's possible to detect which are the main factors affecting the target variable and, possibly, to control them aiming to reduce pollutants effects.
An other thing coming out from this is also that more the sample size is limited, more an accurate selection of the most weighted variables is needed to increase its performance.
Therefore, feature selection should improve consistently prediction results if the size of the sample is small.
For sure models like that could be helpful for the implementation of precise forecast model. Indeed, one usage should be to make average through the an ensemble technique using these explainable models as an important component.
\break
In this work so it's highlighted the effect of how the training in ML should benefit from an accurate selection of variable. 
Instead of faultless building model with exact predictions, the results in this research was pointed more to the fact that model was explainable from the covariates chosen. 
One of the future outcomes from this is absolutely the importance of a model sufficiently explained.
Future research on AI might extend the explanations and interpretability of ML models.
In high-risk applications, AI shouldn't be in blind. 
It's needed to dissect a model for a proper comprehension and explanation before it produce any outputs, especially with large data set.
Performance grows up as the model complexity, turning such systems into “black box” approaches and implying uncertainty in
the way they work and come to decisions. This become a problematic challenge for machine learning systems to be used in critical domains, such as healthcare or economic aspect.
The use of it supports also more efficient debugging for  causalities and a more trust in the model.
For doing this, feature selection attempt to clarify a model’s decision by determine the influence of each input variable. 
These scores alone may not always represent a comprehensive explanation, but it's helpful for understanding the model’s reasoning.