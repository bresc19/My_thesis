
\documentclass{Configuration_Files/PoliMi3i_thesis}

\usepackage{parskip} % For paragraph layout
\usepackage{setspace} % For using single or double spacing
\usepackage{emptypage} % To insert empty pages
\usepackage{multicol} % To write in multiple columns (executive summary)
\setlength\columnsep{15pt} % Column separation in executive summary
\setlength\parindent{0pt} % Indentation
\raggedbottom  

% PACKAGES FOR TITLES
\usepackage{titlesec}
% \titlespacing{\section}{left spacing}{before spacing}{after spacing}
\titlespacing{\section}{0pt}{3.3ex}{2ex}
\titlespacing{\subsection}{0pt}{3.3ex}{1.65ex}
\titlespacing{\subsubsection}{0pt}{3.3ex}{1ex}
\usepackage{color}

% PACKAGES FOR LANGUAGE AND FONT
\usepackage[english]{babel} % The document is in English  
\usepackage[utf8]{inputenc} % UTF8 encoding
\usepackage[T1]{fontenc} % Font encoding
\usepackage[11pt]{moresize} % Big fonts

% PACKAGES FOR IMAGES
\usepackage{graphicx}
\usepackage{transparent} % Enables transparent images
\usepackage{eso-pic} % For the background picture on the title page
\usepackage{subfig} % Numbered and caption subfigures using \subfloat.
\usepackage{tikz} % A package for high-quality hand-made figures.
\usetikzlibrary{}
\graphicspath{{./images/}} % Directory of the images
\usepackage{caption} % Coloured captions
\usepackage{xcolor} % Coloured captions
\usepackage{amsthm,thmtools,xcolor} % Coloured "Theorem"

% STANDARD MATH PACKAGES
\usepackage{amsmath}
\usepackage{amsthm}
\usepackage{amssymb}
\usepackage{amsfonts}
\usepackage{bm}
\usepackage[overload]{empheq} % For braced-style systems of equations.
\usepackage{fix-cm} % To override original LaTeX restrictions on sizes

% PACKAGES FOR TABLES
\usepackage{tabularx}
\usepackage{longtable} % Tables that can span several pages
\usepackage{colortbl}

% PACKAGES FOR ALGORITHMS (PSEUDO-CODE)
\usepackage{algorithm}
\usepackage{algorithmic}

% PACKAGES FOR REFERENCES & BIBLIOGRAPHY
\usepackage[colorlinks=true,linkcolor=black,anchorcolor=black,citecolor=black,filecolor=black,menucolor=black,runcolor=black,urlcolor=black]{hyperref} % Adds clickable links at references
\usepackage{cleveref}
\usepackage[square, numbers, sort&compress]{natbib} % Square brackets, citing references with numbers, citations sorted by appearance in the text and compressed
\bibliographystyle{abbrvnat} % You may use a different style adapted to your field

% OTHER PACKAGES
\usepackage{geometry}
\usepackage{lmodern}
\usepackage[english]{babel}
\usepackage{blindtext}
\usepackage{microtype}
\usepackage{longtable}
\usepackage{multirow}
\usepackage{graphicx}
\usepackage{threeparttable}
\newsavebox\mysavebox
\usepackage{xcolor,comment, subfiles, graphicx, caption, longtable, subfig, fancyhdr}
\usepackage{booktabs}
\usepackage{notoccite}
\usepackage{chemist}
\usepackage{afterpage}

\usepackage{pdfpages} % To include a pdf file
\usepackage{afterpage}
\usepackage{lipsum} % DUMMY PACKAGE
\usepackage{fancyhdr} % For the headers
\fancyhf{}

% Input of configuration file. Do not change config.tex file unless you really know what you are doing. 
\input{Configuration_Files/config}

\newcommand{\bea}{\begin{eqnarray}} % Shortcut for equation arrays
\newcommand{\eea}{\end{eqnarray}}
\newcommand{\e}[1]{\times 10^{#1}}  % Powers of 10 notation

%----------------------------------------------------------------------------
%	ADD YOUR PACKAGES (be careful of package interaction)
%----------------------------------------------------------------------------

%----------------------------------------------------------------------------
%	ADD YOUR DEFINITIONS AND COMMANDS (be careful of existing commands)
%----------------------------------------------------------------------------

%----------------------------------------------------------------------------
%	BEGIN OF YOUR DOCUMENT
%----------------------------------------------------------------------------

\begin{document}

\fancypagestyle{plain}{%
\fancyhf{} % Clear all header and footer fields
\fancyhead[RO,RE]{\thepage} %RO=right odd, RE=right even
\renewcommand{\headrulewidth}{0pt}
\renewcommand{\footrulewidth}{0pt}}

%----------------------------------------------------------------------------
%	TITLE PAGE
%----------------------------------------------------------------------------

\pagestyle{empty} % No page numbers
\frontmatter % Use roman page numbering style (i, ii, iii, iv...) for the preamble pages

\puttitle{
	title=Feature Selection for Explainable Machine Learning Models, % Title of the thesis
	name=Matteo Bresciani, % Author Name and Surname
	course=Geoinformatics Engineering - Ingegneria Geoinformatica, % Study Programme (in Italian)
	ID  = 944639,  % Student ID number (numero di matricola)
	advisor= Prof.Maria Antonia Brovelli, % Supervisor name
	coadvisor={Name Surname, Name Surname}, % Co-Supervisor name, remove this line if there is none
	academicyear={2022-23},  % Academic Year
} % These info will be put into your Title page 

%----------------------------------------------------------------------------
%	PREAMBLE PAGES: ABSTRACT (inglese e italiano), EXECUTIVE SUMMARY
%----------------------------------------------------------------------------
\startpreamble
\setcounter{page}{1} % Set page counter to 1

% ABSTRACT IN ENGLISH
\chapter{Abstract} 
The arrival of big data from heterogeneous sources has had an enormous impact on pollution analysis.\\
At the same time, techniques such as Machine Learning models become more and more the main approaches to make predictions. But with data sets with large samples and covariates, AI models could be more complex to interpret and explain.\par
One machine learning issue is the fact that models are usually a kind of “black box”: systems that evaluate output are not fully understandable by humans. \\
It should be important to interpret the prediction, especially for critical decisions such as healthcare diagnostics.
A way to interpret models for better reliability is to compare feature importance: giving a hierarchy of features could be very useful to distinguish the most influential variables from the ones which could confound model decisions.\par
This thesis wants to contribute to the preprocessing phase that is performed for \gls{ml} models, considering it essential before model training.\\
The challenge of this work is to propose an approach based on the concept of finding the best potential predictive variables with feature selection.\\
The advantage of using it is that the selected feature helps reduce redundancy and increase the interpretability of subsequent predictive models.
In detail, this thesis presents a set of tools to preprocess data and select variables with filter, wrapper, and embedded feature selection methods.\par
In this thesis, this approach has been applied in a case study of D-DUST project, which aims to analyze the impact of intense agriculture on air pollution in Lombardy.\\
Some ML models are also developed to compare the contribution to the accuracy and interpretability of the results.\\
The experimental results obtained by tests with different configurations and resolutions are analyzed and compared using analogous scientific literature.
\\
\\
\\
\\
\textbf{Keywords:} here, the keywords, of your thesis % Keywords

% ABSTRACT IN ITALIAN
\chapter*{Abstract in lingua italiana}
Qui va l'Abstract in lingua italiana della tesi seguito dalla lista di parole chiave.
\\
\\
\textbf{Parole chiave:} qui, vanno, le parole chiave, della tesi % Keywords (italian)

%----------------------------------------------------------------------------
%	LIST OF CONTENTS/FIGURES/TABLES/SYMBOLS
%----------------------------------------------------------------------------

% TABLE OF CONTENTS
\thispagestyle{empty}
\tableofcontents % Table of contents 
\thispagestyle{empty}
\cleardoublepage
\mainmatter % Begin numeric (1,2,3...) page numbering

% --------------------------------------------------------------------------
% NUMBERED CHAPTERS % Regular chapters following
% --------------------------------------------------------------------------
\chapter{Introduction}
Today, we have to deal with new technologies such as the internet, GPS or satellite platforms, which provide us numerous and large samples of data.\\ 
The data growth in the last decades implies a demand for not only a better collection and storage of data but also a way to extract relevant information and discard eventually the ones which are useless, dirty or wrong.\par
In this scenario, a set of techniques of preprocessing such as data cleaning and transforming is required in order to extract knowledge from vast datasets \cite{garcia2016big}.
\par
For example, in the last years, exponential growth of spatial data occurred thanks to devices and instrumentation such as satellite platforms and ground sensor stations. \\
One usage is to address them towards sustainable development activities through models and tools in order to improve our environment. \\
Indeed, Big Data is having a crucial relevance in reaching the target of United Nations’ Sustainable Development Goals  \cite{zhang2019orchestrating}.
One critical phenomenon that puts risks on the 'Good health and well-being' and 'Climate action' goals is air pollution.\\
Indeed, air pollution is nowadays considered one of the world's largest environmental health risk \cite{fuller2022pollution}.\\
Science demonstrated that deterioration of ambient air quality, due to the growing concentration of pollutants in the atmosphere, has caused a significant increment of deaths in the world.\par  
Pollutants such as particulate matter, ozone, carbon monoxide and ammonia cause respiratory diseases and are important sources of mortality.
Almost the entire global population (99\%) breathes air that exceeds WHO air quality limits and threatens their health \cite{WHOreport}.\newline
In Europe, the air is becoming cleaner, but persistent pollution, especially in cities, is damaging the health of people. One of the last reports, based on the European Environment Agency (EEA), shows that exposure to air pollution caused around 500,000 early deaths in the European Union (EU) in 2018  \cite{european2018air}.\par
One of the most harmful pollutants is \gls{pm}, which can penetrate your lungs or even your bloodstream.\newline 
Particle pollution includes PM10 and PM.25, with less than, respectively, 2.5 and 10 micrometers diameters.
Most particles come from other contaminants such as sulphur dioxide and nitrogen oxides, which are pollutants emitted from power plants, industries and automobiles.\par
However, a significant source of PM is one generated by intensive farming \cite{burkart2007diffuse}.
In particular, this is a relevant issue in the Po Valley, where intensive agricultural activity is highly employed.\\
In this context, human civilization is trying to limit pollution and improve the environment with the use of technology.\newline
Technology is helping to clean up air pollution, with data-based solutions helping make our cities healthier places to live.\newline
Monitoring, analyzing and predicting air quality in urban areas is one of the tools to cope with the climate change problem.\par
The advent of modern Artificial Intelligence (AI) techniques such as \gls{ml} can be considered as new possibilities for researchers to find solutions to various problems affecting air quality and climate change.
\\  
Data have to be coded so that they may be easily parsed by the machine. 
Indeed, data in the real world is often dirty with inconsistencies, noise, and missing values since are aggregated from different sources. So it's important to improve the data quality by removing redundant and wrong pieces of data.\\
In addition to data cleaning, it is essential that the data used for the training in \acrshort{ml} models are appropriate, by discarding eventual confounding or improper data.\\
A feature selection is required for this purpose, since helps to choose the most considerable variables.\par
My work is focused in detail on this last step, in which I also tried to interpret, by seeing the results, how each factor affects the target variable, which in my case study describes pollution phenomena related to agriculture.\\
The following case study is part of the \gls{d-dust} project which aims to detect factors that contribute more to agricultural pollution (PM2.5 and ammonia) with also a reasonable explanation from the literature.\\
The D-DUST project, funded by Fondazione Cariplo’s ‘Data Science for Science and Society’ call for proposals, counts on Politecnico di Milano, \gls{dica} as the lead partner.\newline
D-DUST aims to provide information on the impact of agricultural and livestock activities on pollutants in the Po Valley (North Italy).\\
Data from ground sensors are combined with observations provided by satellite platforms and, using data science techniques such as machine learning and geostatistical models, support the monitoring of farming-related PM.\\
Indeed, D-DUST aims to verify the impact of this integration for PM monitoring and prediction. 
The merge of data from these two different sources could help the implementation of more accurate predictive models, thanks to the sampling accuracy of the ground sensors and the granularity of the satellite observations.\\
Combination of traditional measurements from ground sensors with the satellites observations is considered nowadays a new monitoring approach \cite{de2018modelling}.\\
The last target of the project is to provide data-driven best-practices to policymakers, farming operators and citizens in order to minimize the production processes' effects on air quality.\par
In this thesis, the approach applied to this case study by means of the selection of the most remarkable covariates that impact pollutants (such as PM2.5 and NH3) will contribute in D-DUST for modelling phase.\\
In the next chapter of my report, I will show the tools developed and the strategy chosen to reach this goal. 
This is the content of the next chapters:

\begin{itemize}
  \item \autoref{chap:background} (\textbf{Background and state of art}): it describes the scenario in which my thesis work takes place, referring in particular to the state of the art;
  \item \autoref{chap:Overview} (\textbf{Overview}): it shows the main steps I take in my work;
  \item \autoref{chap:case} (\textbf{Case Study and Data Modelling}): it's focused on the case study by explaining each step taken in detail, for both the feature selection and modelling part;
 \item \autoref{chap:res} (\textbf{Interpretation of the results}): it's focused on the results achieved in the case study (both feature selection and models);
 \item \autoref{chap:conclusion} (\textbf{Conclusions}): it summarises the findings through the results obtained and aims to display future opportunities.  
\end{itemize} 

\chapter{Background and state of art}
\label{chap:background}
In this chapter, I'm going to contextualize the state of the art of my research work. 
Besides, I'll explain the target to reach and the solution applied.\par
Nowadays lots of businesses use Big Data for analysing and providing actionable knowledge.\\
Big Data is defined as large data sets collected from different sources such as applications, social networks and websites.\\
For this, we are seeing nowadays a rapid growth of them, which causes a rise of redundancy and corrupted data.\\
The next consequence would be an accuracy reduction if models are built from them.\\
Before developing a predictive model, Feature Selection is a necessary step to reduce the number of input variables.
Data should be pre-processed with FS before training, to reduce overfitting and improve accuracy.
\newline
Nowadays, with the large volume and variety in Big Data, FS is increasingly becoming an essential preprocessing step in machine learning algorithms \cite{kamolov2021feature}.\\
It is desirable to both reduce the computational cost of modelling and, in some cases, to improve the performance of the model.\par
Several application domains have been reported in the literature in which FS helps to solve these issues, in particular for classification tasks.
The following article \cite{forman2003extensive} has reported how FS in text mining problems contribute to discarding words with only a limited occurrence using filter methods.\\
Another domain is image processing. For example, this study  \cite{muvstra2012breast} shows us how wrapper methods are used to select the best subset of extracted features to improve accuracy in breast density classification.\\
Also in bioinformatics field is used, since could help to identify the most discriminant genes in the classification of distinguishing between healthy and tumour tissues \cite{dessi2013comparative}.\\
Feature Selection is also used before a prediction process because it helps to increase performance to get rid of irrelevant or redundant variables \cite{bagherzadeh2016tutorial}.\\  
For instance, FS was also employed in the environmental field to predict and monitor air pollution \cite{ul2022improving}, which is also one of the goals of D-DUST. 
\par
D-DUST aims to predict primary pollutants from intensive farming.\\ 
To contribute to this, I performed FS on variables from different kinds of data resources used.
The physical and chemical variables chosen are the ones that, according to the scientific literature review, are most associated with the formation of primary and secondary atmospheric PM. 
Data include satellite observations (as Sentinel-5P) that can provide high spatial and temporal resolution of air pollutants variables. \\
One purpose of D-DUST is also to combine satellite-based information with ground sensor observation and air quality/atmospheric models. Indeed, satellite observations can give us useful information, especially in rural areas where ground sensors are limited thanks to their granularity.\\
Ground monitoring stations provide air quality and meteorological data, by ensuring also the validation of satellite observations.
The most commonly used by D-DUST and this work are the ones offered by ARPA stations.
Models included are the ones that provide a forecast of atmospheric and air quality such as the \gls{cams} models.
Other data processed with FS include time-invariant territorial features such as population density, land use or vegetation indexes since are correlated to the presence of pollutants.
Information more detailed regarding the data used are collected in the case study chapter \ref{chap:case}.
\par
Feature Selection not help only to make better model performance, but also to have a better comprehension of the data that are used by ML models during training.\\ 
Indeed, an aspect taken into consideration in this work is also that a \acrshort{ml} model trained with so many features would be a black box, in which a lack of interpretability could not be able to explain the decisions taken by the \gls{aii} algorithms.\\
So it's needed to care about interpretability to discard eventually confounding variables which can suggest there is a correlation when in fact there is not, even if the model's accuracy is extremely high.\\
For example, a new paper by Alex DeGrave et al. \cite{degrave2021ai} shows that a Deep Learning model trained with improper data was taking shortcuts in COVID-19 detection on radiographs due to the position of certain markers rather than on the actual radiograph.
Another common example of this is the confounding correlation between the number of shark attacks and ice cream sales. 
The following picture (Figure \ref{fig:shark}) \cite{shark-icecream} highlights how there's no direct relationship between shark attacks and ice cream sales. Instead, they're both caused by a third factor (High temperature).\newline
\begin{figure}[H]
    \centering
    \includegraphics[scale=0.25]{images/confounding.png}
    \caption{Representation of the relationship between shark attack and ice-cream sales.}
    \label{fig:shark}
\end{figure}
Therefore, the key to increasing the interpretability of a given model is to wonder if a given factor should drive the final decision.\par
In this context, in which the black-box nature of \acrshort{ml} algorithms raises ethical and judicial concerns inducing a lack of trust \cite{9141213}, \gls{xai} aims to create a fully interpretable model.\\
Before the advent of XAI, the scientific community was focused on the predictive power of algorithms rather than the understanding behind these predictions.\\
This need for reliable high-performing models led to XAI, a field focused on the interpretation of how AI systems take decisions.
This issue of interpretability and clarity is becoming increasingly significant nowadays. \\
This is consistent with what could be seen in the figures \ref{fig:AI_XAI}.
The following figures show that publications with the words \gls{dl} and Explainable AI increased between 2010 and 2020. 
In addition, the trend of Explainable AI grew up in the last 3 years, while the curve of DL seems to have reached a saturation state.\\
\begin{figure}[H]
\centering
    \includegraphics[scale=0.30]{images/DL-XAI.jpg}
    \caption{Plot of the interest progression of DL and XAI provided by Google Trend in the period between 2010 and 2020 \cite{angelov2021explainable}.}
    \label{fig:AI_XAI}
\end{figure}
In this work, the focus is on model interpretability instead of explainability, even if in literature there are references that describe them in the same way.\\
Interpretable Artificial Intelligence (or Interpretable Machine Learning) helps to understand how \acrshort{ml} algorithms make predictions, with the use of feature selection methods to clarify the model decision.
\par
Feature selection can provide relevant explanations by quantifying the influence of each input variable on the model's output with a score.\\
The contribution of FS will be afterwards considered during the development of ML models by D-DUST.\\
This step aims to provide a weighted score of each environmental variable considering the pollutants emitted by intensive farming activities as target variables (e.g PM2.5 and Ammonia). \\
In this work, scores will be interpreted with findings in the literature to confirm them.\\
Because there is no best feature selection technique, I performed and combined different supervised methods. \\
Then, the most weighted input variables will be used as predictors inside the D-DUST project to monitor primary pollutants from intensive farming with data science techniques, such as ML models.\\
In my work, after the reduction of the variables was made by FS, I also built 2 ML models to predict target variables related to intensive farming activities such as PM2.5 and ammonia.
\bigbreak
According to the above, my work comes in this scenario.
In the next chapter, the FS procedure will be described in detail.
\\


\chapter{Overview}
 \label{chap:Overview}
In this chapter, I illustrate the overview of the pre-processing phase, by explaining in detail each step taken and tool used.
My work is focused on the first phase of a data analysis procedure which is the pre-processing.
Data preprocessing (or data preparation) is the process of transforming raw data into a suitable format for modelling. 
Indeed, raw data is in most cases incomplete and noisy.\\
Today, dealing with a large amount of information, the probability of incorrect data is higher without proper data pre-processing.
Only high-quality data can generate accurate models and predictions. \\
The view and quality of data are very relevant before running any analysis.\\
Therefore, it is crucial to process data with the best possible quality before using them as training samples with artificial intelligence and machine learning predictive models.\par

\section{Data Collection}
Data collection is the process of collecting information on variables of interest to answer relevant questions. \newline
Relevant data are collected from their sources and merged in data structures (such as Dataframes). \\
To do that I was helped by python libraries such as Pandas \cite{pandas}, Geopandas \cite{geopandas} and NumPy \cite{numpy}.
\section{Data Cleaning}
\label{sec:Data cleaning}
Data has to be prepared in accordance with the supervised feature selection.
Data cleaning aims to fix problems or errors in messy data.\\ There are many reasons data may have incorrect values, such as being corrupted, duplicated or invalid. \newline
Data cleaning can be done by removing rows or columns. Alternatively, it might involve replacing the observations with new values. \newline
For doing this, I present this solution in sequence, using methods provided by Pandas library (Pandas.dropna):
\begin{itemize}
\item Drop of samples having target variable with NaN value;
\item Drop of columns (values assumed by each covariate) having at least a NaN value;
\end{itemize}
\subsection{Remove variables with low variance}
An approach to remove columns is to consider the variance of each column variable. The variance is a statistic representing the expected value of the squared deviation from the mean $\mu$ of a given variable X. 
\begin{equation}
  Var(X) = E[(X-\mu)^2]
\end{equation}
The variance can be used as a filter for identifying columns to be removed from a given data set. 
Using a feature with low variance only adds complexity to the feature selection and the prediction.\newline
In order to do that, I performed a method called nearZeroVar (Near zero variance), a method from the \gls{care} package in R.\\
This method finds covariates that have one unique value (i.e. with zero variance) or covariates that have both of the following requirements:
\begin{itemize}
\item they have very few unique values relative to the number of samples (represented by the percentage of unique values);
\item the ratio of the frequency of the most common value to the frequency of the second most common value is large;
\end{itemize}
After this detection, features with low variance should be meaningless and consequently discarded. 
\section{Data Transformation}
Data need to be scaled. As a matter of fact, each feature in our data has varying degrees of magnitude, range, and units. \\This is an issue for machine learning algorithms because they are highly sensitive to these features.\\ 
Having input variables with different units (e.g. ug/m\textsuperscript{3}, °C, hours or mol/m\textsuperscript{2}) implies data at different scales. This could raise the difficulty of the problem being modelled. \newline
Hence, a common scale is needed through normalization or standardization to improve data quality.\newline
Many ML and regression algorithms perform better when the numerical input and output variables are scaled to a common standard range. \newline
For instance, it's proved that neural networks trained with scaled data perform better in terms of \gls{mse}  \cite{shanker1996effect}.\\
In this step, two types of transformation have been done:
\begin{itemize}
    \item Standardization on input data;
    \item Normalization on output data;
\end{itemize}
\subsection{Standardization}
The most common data transformation is to centre and scale each variable value. In order to do that, the average value is removed from all the values. As a result of centring, the predictor will have a zero mean \cite{kuhn2013applied}.\\
Standardization consists of rescaling data following a Gaussian distribution of values with a mean equal to 0 and a standard deviation equal to 1:
\begin{equation}
  Z = \frac{X-\mu}{\sigma}
\end{equation}
\begin{equation}
\mu = \frac{(\sum_{n=1}^{N} X_i)}{N}
\end{equation}
\begin{equation}
\sigma = \sqrt{\frac{\sum_{n=1}^{N} (X_i-\mu)^{2}}{N-1}}
\end{equation}
Where:
\begin{itemize}
\item Z is the numeric value standardized for a given covariate;
\item X is the numeric value to be standardized of a given covariate;
\item $\mu$ is the mean value for the set of values assumed by a given covariate;
\item $\sigma$ is the standard deviation for the set of values assumed by a given covariate;
\end{itemize}
Every term was computed using the Scipy library (scipy.stats). 
\bigbreak
\subsection{Normalization}
Data Normalization is the process of a different method for adjusting data at different scales. Data are scaled in a range between 0 and 1 and was performed only for the feature selection methods output.
Output normalization is an essential step for the comparison of different outputs since data ranges vary for each method used.\newline
This was performed in my notebooks from the scikit-learn library (sklearn.preprocessing) using the MinMaxScaler method.
\section{Feature Selection}
Feature Selection is the core part of this study. It's the process of reducing the number of input variables when developing a predictive model by basing it on a target (or output) variable. \\
Data collected, even if have been cleaned and transformed, are anyway characterized by big amount of redundant variables.\\
Discarding irrelevant data is essential before applying the machine learning model in order to:
\begin{itemize}
\item Reduce Overfitting: less opportunity to make decisions based on noise;
\item Improve Accuracy: less misleading data means that modelling accuracy improves. Predictions can be greatly distorted by redundant attributes;
\item Reduce Training Time: With fewer data, an algorithm will train faster;
\end{itemize}
In this step, which will be explained in detail in the next chapters, the reduced input variables are the ones that are meaningless with respect to a target variable as the output. \newline
Due to the fact that there is no best feature selection technique, many different methods are performed, each of which gives different correlation results.\par
Inside the D-DUST project, this step helps to select of the most important features for the next step of modeling using Machine Learning and geostatistical models. \\
Feature selection allows to determine which variables are most correlated with the selected target variable, such as PM2.5 and ammonia in the context of this project. 
The removal of unrelated and uninfluential variables brings benefits to subsequent modeling.
\par
In the following subsections, each implemented FS method is described in detail, classified into three main categories, as we can find in the literature \cite{stanczyk2015feature}.
\subsection{Filter Methods}
Filter-based feature selection methods adopt statistical measures to evaluate the correlation/dependence between input variables.\newline
In terms of computation, they are very fast and very suitable to remove duplicated, correlated, redundant variables \cite{saeys2007review}. \newline
These methods evaluate each feature individually without considering the interaction between them. Therefore, they do not fit well if the data has high multicollinearity \cite{daoud2017multicollinearity}.\\
Multicollinearity is a phenomenon that occurs if a group of independent variables are statistically correlated between them. 
This correlation issue could represent a problem during regression since they should be independent. 
\subsubsection{Pearson coefficient}
The Pearson coefficient is one of the most widely used indices for measuring linear correlation in statistics. It ranges between -1 and 1, where:
\begin{itemize}
\item 1 indicates a strictly positive correlation;
\item -1 indicates a strictly negative correlation;
\item0 indicates no correlation between the features;
\end{itemize}

Therefore, taking only its absolute value, 1 implies that a linear equation perfectly describes the relationship between X and Y, for both positive and negative correlations. \newline
The Pearson index between an independent variable X and a target variable Y is defined by the following formula:

\begin{equation}
  \rho_{x,y} = \frac{Cov(X,Y)}{\sigma_x\sigma_y}
 \end{equation}

  Where:
  \begin{itemize}
      \item Cov(X,Y) is the covariance between X and Y;
      \item $\sigma_x$ and $\sigma_y$ are the standard deviations of x and y;
  \end{itemize}
  



\subsubsection{Kendall Tau}
Kendall Tau index is used to measure monotonic relationships as a test statistic to determine whether two variables are statistically dependent. \newline
While in linear correlation two variables move together at a constant rate, monotonic or rank correlation measures how likely two variables move in the same direction, but not necessarily in a constant manner. \newline
Like Pearson’s correlation, Kendall’s has a value between -1 and 1, where:

\begin{itemize}
\item -1 represents a strictly negative monotonic relationship;
\item 1 represents a strictly positive monotonic relationship;
\item 0 representing no relationship;
\end{itemize}
Given a sample X and Y with n as sample size, the tau index is computed by the formula:
\begin{equation}
  \tau_{x,y} = \frac{n_c-n_d}{\frac{1}{2}n(n-1)}
\end{equation}
where:
\begin{itemize}
\item n\textsubscript{c} = \# of concordant pairs (concordant pairs: the pairs are ordered in the same way);
\item n\textsubscript{d} = \# of discordant pairs (discordant pairs: the pairs are ordered differently);
\end{itemize}
In order to better explain how n\textsubscript{c} and n\textsubscript{d} are computed I show the following example. \par
Suppose that 2 professors provide (variables labelled as X and Y) a mark to each student (sample). 
Then, for each student i(x\textsubscript{i}, y\textsubscript{i}) I:
\begin{itemize}
    \item sum 1 to n\textsubscript{c} if its pairs of votes (x, y) are concordant to an other student j ((x\textsubscript{i} > x\textsubscript{i} and y\textsubscript{i} > yj) or (x\textsubscript{i} < x\textsubscript{i} and y\textsubscript{i} < y\textsubscript{i}));
    \item sum 1 to n\textsubscript{d} if its pairs of votes (x, y) are discordant to an other j-student (x\textsubscript{i} > x\textsubscript{i} and y\textsubscript{i} < y\textsubscript{i}) or (x\textsubscript{i} < x\textsubscript{i} and y\textsubscript{i} > y\textsubscript{i});
    \item neither if x\textsubscript{i} = x\textsubscript{j} or y\textsubscript{i} = y\textsubscript{j};
\end{itemize} 
In the figure, in the third and fourth columns are counted each concordant or discordant pair compared to the students below.

\begin{figure}[H]
    \centering
    \includegraphics[scale=0.21]{images/k}
    \caption{The following table highlights how n\textsubscript{c} and n\textsubscript{d} are obtained to compute the Kendall Tau coefficient.}
    \label{fig:kendall}
\end{figure}

\subsubsection{Spearman Rho}
Spearman’s index is very similar to Kendall’s. As the previous filter methods, it ranges between -1 and 1, and it's considered less robust than Kendall's.
It is computed as follows:
\begin{equation}
\rho_{x,y} = \frac{6\sum_{n=1}^{N} d_i^2}{n(n-1)^2}
\end{equation}
\begin{itemize}
\item d\textsubscript{i}: difference between each corresponding X\textsubscript{i} and Y\textsubscript{i};
\item n: size of the sample;
\end{itemize}

Finally, as I did for Pearson and Kendall coefficients, I take into consideration only its absolute value to weigh the correlation for each variable in the Feature Selection.

\subsubsection{Fisher Score}
This method returns the score of the variables based on the fisher’s score in descending order. \newline
Its algorithm is implemented using the SelectKBest method from the scikit-learn library (sklearn.feature\_selection).\newline
\begin{equation}
F = \frac{\sigma_x^2}{\sigma_y^2}
\end{equation}
Where:
\begin{itemize}
\item F is the Fisher's scores obtained between the independent (X) e dependent variable (Y);
    \item $\sigma_x^2$ and $\sigma_y^2$ are the variances of X and Y;
\end{itemize}
\bigbreak
\subsection{Wrapper Methods}
Wrapper methods, as the name suggests, wrap a machine learning model, with different subsets of input features. In this way, the subsets are evaluated following the best model performance.
One disadvantage of this approach is the computational costs.\newline
Their execution for many subsets of variables can become unfeasible. 
\subsubsection{Random Forest Importance}
Feature importance is a built-in function of the Random Forest algorithm. It's also called Gini importance (or mean decrease impurity) and is commonly used as the splitting criterion in decision trees problems. \\
It's computed with the mean of impurity decrease applied over all trees. \\ 
Feature selection made with the impurity reduction of splits is increasingly used for its simplicity and velocity to be computed.\\
The scores are evaluated as attributes (feature\_importances\_) of RandomForestRegressor of the scikit-learn library (sklearn.ensemble).
\bigbreak
\subsection{Embedded Methods}
Embedded methods are instead characterized by the benefits of both the wrapper and filter methods, by including interactions of features, but also have a reasonable computational cost.
\subsubsection{Recursive Feature Elimination}
\gls{rfe} is a wrapper feature selection algorithm that also works with filter-based feature selection internally.\newline
It consists in looking for the best subset of features by starting with all features and removing some of them until the desired number remains.\newline
This is computed using \acrshort{rfe} of scikit-learn library (sklearn.feature\_selection).\\
To obtain a score for each variable, I consider whether it is selected or not a value (with support\_ attribute).\\
If the attribute is selected will be equal to 1, otherwise to 0.\\
The number of selected variables, as the default option of the method, is half the total number of covariates.
\pagebreak
\subsection{Borda Count: averaging FS results}
One of the most important challenges in this study is the lack of a universal feature selection method that produces results that are common with all the FS techniques. Choosing a feature selection method from a vast range of choices can be challenging. \newline
So it needs an ensemble technique that aims to make it more robust across various algorithms. In this work, we adopt the ensemble approach described in this study \cite{sarkar2014robust}, using the Borda Count algorithm. Initially, Borda Count was a voting system method named for Jean-Charles de Borda \cite{borda1784memoire}.\newline
Borda Count is used as a rank-based combination technique used to evaluate an average score for each feature. \\In this method, assuming that each score evaluated by each FS method are sorted in descending order, points are assigned to candidates (variables) based on their ranking; 1 point for the last choice (the most meaningless by its score), 2 points for second-to-last, and so on. \\Finally, the points for all the ballots are summed up and the candidates with the highest total of points are the winners (the features with the largest points are the most significant).
\bigskip
\section{Model prediction}
Prediction is a type of analysis that uses techniques and tools to build predictive models and predict outcomes. \\
In my work, predictive analysis is performed to make prediction on the target with data processed in the first phase as input.\newline
Model predictions are implemented through regression analysis, used to estimate the relationships between a dependent variable and one or more independent variables.\\
In particular, I used supervised techniques based on Machine Learning where the model built is fit with the training data set and its performance evaluated through the test set. 
\\
At this point variables with the highest number of votes can be used as input in ML models and taken into consideration as the most meaningful factor affecting the target variable.
\par
After the modelling step, an evaluation of the performance predictions is performed in terms of error and accuracy using k-fold cross-validation.\newline
K-Fold cross-validation consists in dividing a data set into k multiple training and validations sets (folds), to improve model results against the random selection of only one training and validation set. Indeed, errors and accuracy evaluation are averaged along the k different random samples.\\
Metrics chosen for evaluation are:
\begin{itemize}
    \item \acrshort{mae} (Mean Absolute Error): it's the mean absolute distance for the observations;
    \item \acrshort{mse} (Mean Squared Error): it measures as MAE the distance of errors from the observations but with the difference of squaring the distance. In this way higher errors weigh more;
    \item \acrshort{rmse} (Root Mean Squared Error): it's computed as the previous term (MSE), with the difference that the square root is applied at the end. RMSE is used because due to the same unit of the target variable.
    The RMSE is always larger or equal to the MAE; the larger is the gap between them, the grater will be the variance in the individual errors of the sample.
    \item R\textsuperscript{2} (Coefficient of determination): it is a statistical index that represents the percentage whereas the target variable is explained in a regression model;
\end{itemize} 
Their formulas are the following:
\subsubsection{Mean Absolute Error}
\begin{equation}
MAE = \frac{1}{N}\sum_{i=1}^{N}|y_i-\hat{y}_i|
\end{equation}
Where:
\begin{itemize}
    \item N is the number of the samples;
    \item $\hat{y}$\textsubscript{i} is the i-th samples predicted;
    \item $y$\textsubscript{i} is the i-th sample of the test set used for validation;
\end{itemize}
\pagebreak
\subsubsection{Mean Squared Error}
\begin{equation}
MSE = \frac{1}{N}\sum_{i=1}^{N}(y_i-\hat{y}_i)^2
\end{equation}
Where:
\begin{itemize}
    \item N is the number of the samples;
    \item $\hat{y}$\textsubscript{i} is the i-th samples predicted;
    \item $y$\textsubscript{i} is the i-th sample of the test set used for validation;
\end{itemize}
\subsubsection{Root Mean Squared Error}
\begin{equation}
RMSE = \sqrt{MSE}= \sqrt{\frac{1}{N}\sum_{i=1}^{N}(y_i-\hat{y}_i)^2}
\end{equation}
\subsubsection{R\textsuperscript{2} score}
\begin{equation}
R^2 = 1 - \frac{RSS}{TSS}    
\end{equation}
Where:
\begin{equation}RSS = \sum_{i=1}^{N}(y_i-\hat{y}_i)^2 \end{equation}is the residual sum of squares;
\begin{equation} TSS =  \sum_{i=1}^{N}(y_i-\bar{y})\end{equation} is the total sum of squares;
\begin{itemize}
    \item $\hat{y}$\textsubscript{i} is the i-th samples predicted;
    \item $y$\textsubscript{i} is the i-th sample of the test set used for validation;
    \item $\bar{y}$ it's mean value of the test set;
\end{itemize}
\bigbreak\bigbreak
In my work, I wasn't focused in detail on the configuration for an optimal ML model. The aim of this phase is only to give an approximate evaluation of how the features selected impact the models' performance. \\
For doing that I implemented 2 different ML supervised models.\\
By literature, we know that interpretability is usually related to a trade-off with accuracy (figure \ref{fig:trade-off}).\par
Since highly accurate algorithms are often less interpretable, in order to detect the impact of feature selection on accuracy Neural Network and Random Forest algorithms are chosen. 
\begin{figure}[H]
    \centering
    \includegraphics[scale=1.4]{images/interpretability_accuracy_tradeoff.jpg}
    \caption{The plot represented in this figure wants to highlight how different ML models are related to the trade-off between accuracy and intepretability  \cite{morocho2019machine}.The more a model is accurate such as Neural Network and Random Forest, the more it's less interpretable due to its "black-box" nature.}
    \label{fig:trade-off}
\end{figure}
\subsection{Neural Network with Keras}
It's one of the deep learning algorithms which is based on the structure of the neural networks, where nodes represent neurons.
Nodes are connected to each other through the layers. \\
In this way, every neuron of a layer is connected to neurons in the next layer.
The structure of my Neural Network is formed by 3 layers:
\begin{itemize}
    \item Input layer: Each node takes the initial data into the network and propagates information to the following layers;
    \item Output layer: It contains the result of the problem. 
    \item Hidden layer: It's actually responsible for the performance and complexity of neural networks. It's placed between the output and input layers;
\end{itemize}
During the training phase, in which the network "learns", independent variables are used as input and processed through weighted associations in which at each step produce results that are compared to the target output.\\ The error computed between them is successively adjusted and updated following the learning rules. The achievement obtained is a result increasingly similar to the values assumed by the target variable.\\ 
For implementing it, I use Keras API from TensorFlow library.


\subsection{Machine Learning with Random Forest}
Random forests are used in regression problems, using a decision tree structure to make predictions. Decision trees answer sequential questions through the routes tracked by trees.\\
The algorithm consists in building a certain number of random decision trees from the bootstrap sample of the original data set (this phase is called bootstrap sampling). Then, the prediction of a certain sample is performed by following each decision tree. \\In a regression problem, the final decision will be an averaged value from the ones obtained by each tree (phase of aggregation).\\
To implement it, I use RandomForestRegressor class from sklearn library. 


\begin{figure}[H]
    \centering
    \includegraphics[scale=0.21]{images/overview.png}
    \caption{The following diagram gives a general explanation of the procedure taken in this thesis. It starts with the pre-processing of the data (collection, cleaning and transformation) and continues with the selection of the most meaningful variables (feature selection). \\
    After that, ML models are implemented and trained with the covariates chosen in the previous step. \\
    Finally, an evaluation of the error metrics based on the validation is performed.}
    \label{fig:overview}
\end{figure}


\section{Notebook implemented}
For processing and analyzing data, I implemented tools collected in Python Notebooks, each one available in the D-DUST repository\footnote{\url{https://github.com/opengeolab/D-DUST/tree/thesis_MB}}.\newline
Its essential steps are shown in figure \ref{fig:overview}.

\subsection{Computation and view of feature selection results}
In order to manage its configuration and the results obtained, a notebook\footnote{\url{https://github.com/opengeolab/D-DUST/blob/thesis_MB/notebooks/fs_results.ipynb}} with a simple user interface is built using the ipywidgets package.
In this interface there are 2 sections:
\begin{itemize}
\item Feature Selection scores: they are graphically shown using multiple barplots, one for each data set previously selected. Barplot are implemented with the use of Plotly library; 
\item Options: in this box is possible to configure the feature selection input:
\begin{itemize}
\item target variable. The usable variables are those coming from ARPA ground sensors;
\item apply (optional) low variance filter for discarding meaningless variable before FS;
\end{itemize}
and the output:
\begin{itemize}
\item choice of method for visualizing its scores;
\item results normalization (optional);
\item order of the scores by descending order or by labels;
\item scale of Y-axis (regular or logarithmic);
\end{itemize}
\end{itemize}
A pseudo-code of how the notebook works is attached in the next page.\pagebreak
\begin{verbatim}
for each configuration in configurations
    for each period in periods
        #Data acquisition
        grid = input(period)
        data_cleaning(grid)
        grid = quasi_zeroVar(X)
        X, Y = get_variables(grid, target_variable)
        #Feature selection phase
        results = compute_feature_selection(X, Y)
    results = borda_count_algorithm(results)
    #results are stored externally in .csv files
    export_tocsv(results) 
    #results exported will have the feature ordered with respect
    #the score obtained with Borda Count algorithm
\end{verbatim}
\begin{figure}[H]
    \centering
    \includegraphics[scale=0.40]{images/notebook.png}
    \caption{Overview of the notebook implemented for FS procedure.In the notebook, there are also cells which aim to export FS results.}
    \label{fig:notebook}
\end{figure}
\pagebreak


\subsection{Computation and view of ML models results}
For build, I use two different notebooks for the evaluation of the Neural Network (link\footnote{\url{https://github.com/opengeolab/D-DUST/blob/thesis_MB/notebooks/Keras_prediction_model.ipynb}}) and the Random Forest model
the Neural Network (link\footnote{\url{https://github.com/opengeolab/D-DUST/blob/thesis_MB/notebooks/RandomForest_prediction_model.ipynb}}). 
It is possible to set these parameters before running the model:
\begin{itemize}
    \item NUMBER\_OF\_COVARIATES: It's the number of the n features with the highest Borda Count score taken as input for the model;
    \item TARGET: It represents the target variable to be predicted by the model;
\end{itemize}
\begin{verbatim}
for each configuration in configurations
    results = []
    for each period in periods
        #Data acquisition
        grid = input(period)
        grid = buffer_knn(grid)
        data_cleaning(grid)
        X, Y = get_variables(grid, TARGET)
        X = get_n_columns(NUMBER_OF_COVARIATES)
        #Modelling in which training and validation are performed using k-fold
        model = new()
        model.training(X, Y)
        res = model.validation(X, Y)
        results.append(res)
    
    #results are stored externally in .csv files
    export(avg(results1))
    export(avg(results2))
 
\end{verbatim}

Each of them imports the feature selected from the previous notebook and exports the accuracy evaluation of the k-fold cross-validation externally. \par
The results are subsequently opened and shown with
this other notebook\footnote{\url{https://github.com/opengeolab/D-DUST/blob/thesis_MB/notebooks/model.ipynb}} through the use of widgets (figure \ref{fig:view}).
\begin{figure}[H] 
    \centering
    \subfloat[Drop-down widgets.]{%
        \includegraphics[width=0.5\textwidth]{images/dropdown.png}%
        %
        }%
    \hfill%
    \subfloat[Table with the results selected.]{%
        \includegraphics[width=0.5\textwidth]{images/table.png}%
        %
        }%
    \caption{In these images is illustrated how the notebook view for viewing the validation results works. \\
    It is sufficient to select the target variable, resolution and configuration desired and with an interactive option provided by ipywidgets library the results are provided through a table.}
    \label{fig:view}
\end{figure}

An overview of the order in which the different notebooks are used in my work thesis is illustrated in Figure \ref{fig:notebooks}.\par
In the next chapters, each step will be described in detail about the procedures adapted for the study case and the results obtained.
\begin{figure}[H]
    \centering
    \includegraphics[scale=0.40]{images/overview _notebooks.png}
    \caption{The following diagram gives a general explanation of the sequence of the notebooks which are used. It starts with fs\_results.ipynb which aims to collect, clean, transform and select the most meaningful variables through the Feature Selection). \\
    After that, Keras\_prediction\_model.ipynb and RandomForest\_prediction\_model.ipynb are used to implement ML models which are trained with the covariates chosen in the previous step. \\
    Finally, the results of the error metrics based on the validation are provided by model.ipynb. }
    \label{fig:notebooks}
\end{figure} 

\chapter{Case of Study and Data Modelling }
 \label{chap:case}
In this chapter, the case study inside the D-DUST project is explained, paying attention to the data used in the different periods, resolutions and target variables.
\section{Data sets description}
This case study aims to discover the main factors which affects mostly the target variable chosen. 
Variables selected are the physical and chemical factors that are most associated to the formation of primary and secondary pollutants. \newline
Therefore, the variables are categorized in 4 different labels:
\begin{itemize}
\item Weather: these elements, such as wind speed and direction, precipitation and air temperature, change in the epochs and can influence air pollution;
\item Pollutant: these variables represent primary and secondary pollutants related to the greenhouse effect;
\item Soil and Vegetation: since soil and vegetation degradation are global concerns and can influence air propagation in the environment, data related to soil and vegetation characteristics are collected;
\item GIS (time-invariant layers): these time-invariant layers are considered to be changeless in the time range considered. Differently from the other types which need constant monitoring, these variables are updated yearly with a lower frequency than the others;
\end{itemize}
The data chosen are open source and are regularly available.
In this phase, data have been collected (in this repository\footnote{ \url{https://docs.google.com/spreadsheets/d/1-5pwMSc1QlFyC8iIaA-l1fWhWtpqVio2/edit\#gid=91313358}}) in grids from different sources and provided in geopackages. 
\subsection{Source types}
In order to better distinguish the characteristics of the data sources, the selected variables are labelled with four different types of sources:
\begin{itemize}

\item Ground Sensor: One of the aims of this case study related to the D-DUST project, is to detect the impact on estimates combining data coming from ground sensors to satellite data, and analyse how the local monitoring could improve.  
Each ground monitoring station which provides meteorological and air quality data belongs mainly to: 
\begin{itemize}
    \item ARPA (Agenzia Regionale per la Prevenzione dell'Ambiente);
    \item ESA Air Quality Platform (low-cost sensor) \cite{esasensor};
\end{itemize} 
\begin{comment}
Even if ground sensor measurament has been already validated by ARPA, there were still values out-of-scale. 
In order to manage this, a supplementary outlier deletion was performed by removing the values that had the absolute value of z score greater than or equal to 4.\\ 
\end{comment}
\item Model: data are estimated through a model built using satellite and meteorological and air quality data as input, such as European data provided by \acrshort{cams};
\item Map layer: this data are time-invariant and are related to Lombardy morphology such as density of roads, population or land use; 
\item Satellite Sensor: They provide data from air quality observation mainly. Satellites providers are Sentinel-5P \acrshort{tropomi} and Terra \& Aqua \acrshort{modis};
\end{itemize}
\bigbreak

In the next lines, each variable is provided in tables, by showing its type, name and description.
\pagebreak
\subsubsection{Meteo Variable (Table)}
\begin{center}
\setlength{\arrayrulewidth}{1.5pt}

\begin{longtable}{ |p{2.3cm}|p{1.8cm}|p{2.3cm}|p{4cm}|p{1cm}|p{2.3cm}| } 
\hline
\textbf{Physical variable} & \textbf{Source type}  & \textbf{Variable name}  & \textbf{Description}  & \textbf{Unit}  & \textbf{Source}\\ 
\hline
\multirow{3}{4em}{Temperature} & Model  & \underline{temp\_2m} & Mean air temperature at 2 m above the land surface.\par & °K & ERA5-Land hourly data.\\ 
& Ground \newline Sensor  & \underline{temp\_lcs} &  Mean air temperature ground measurement - Low Cost Sensor ESA monitoring stations.\par & °C & ESA Air Quality Platform.\\ 
& Ground \newline Sensor  & \underline{temp\_st} &  Mean temperature - ARPA monitoring stations.\par & °C & ARPA \newline Lombardia.\\ \hline

\multirow{4}{4em}{Wind} & Model  & \underline{e\_wind} & Mean eastward wind component 10 m above the land surface.\par & m/s & ERA5-Land hourly data.\\ 
& Ground \newline Sensor  & \underline{n\_wind} &  Mean northward wind component 10 m above the land surface.\par & m/s & ERA5-Land hourly data.\\
& Ground \newline Sensor  & \underline{wind\_speed\_st} &  Mean wind speed on ground  - ARPA monitoring stations. \par& m/s& ARPA \newline Lombardia.\\ \hline
\pagebreak
\hline
\multirow{2}{4em}{Precipitation} & Model  & \underline{prec} & Mean accumulated liquid and frozen water, including rain and snow, that falls to the Earth's surface. It is the sum of large-scale precipitation. \par & mm & ERA5-Land hourly data.\\ 
& Ground \newline Sensor  & \underline{prec\_st} &  Mean precipitation in each cell in the time range - ARPA monitor stations. \par & mm & ARPA \newline Lombardia.\\ \hline

\multirow{2}{4em}{Air Humidity} & Ground \newline Sensor  & \underline{air\_hum\_st} & Mean air moisture measurement in the time range - ARPA monitoring stations.\par & \% & ARPA \newline Lombardia.\\ 
& Ground \newline Sensor  & \underline{air\_hum\_lcs} &  Mean air moisture ground measurement - Low Cost Sensor ESA monitoring stations.\par & \% & ESA Air Quality Platform.\\ \hline

\multirow{1}{4em}{Air Pressure} & Model   & \underline{press} & Mean weight of all air in a column vertically above the area of the Earth's surface represented at a fixed point.\par & Pa & ERA5-Land hourly data.\\ \hline

\multirow{1}{4em}{Solar Radiation} & Ground \newline Sensor  & \underline{press} & Global radiation measurement - ARPA monitoring station.\par & W/m\textsuperscript{2} & ARPA \newline Lombardia.\\ \hline

\hline
\caption{Table of Meteorological variables.}

\end{longtable}
\end{center}

\subsubsection{Pollutants Variables (Table)}


\begin{center}
\setlength{\arrayrulewidth}{1.5pt}
\begin{longtable}{ |p{1.5cm}|p{1.5cm}|p{2.3cm}|p{4cm}|p{2cm}|p{2.3cm}| } 
\hline
\textbf{Physical variable} & \textbf{Source type}  & \textbf{Variable name}  & \textbf{Description}  & \textbf{Unit}  & \textbf{Source}\\ 
\hline
\multirow{1}{4em}{Dust} & Model  & \underline{dust} & Mean dust concentration at 0m level provided by CAMS (Ensemble Median - Analysis).\par & $\mu$g/m\textsuperscript{3} & CAMS Model.\\ \hline

\multirow{3}{4em}{AOD} & Satellite \newline Sensor  & \underline{aod\_055} & Mean Aerosol Optical Depth at 550nm.\par & dimension-\newline less & MODIS Terra+Aqua.\\ 
& Satellite \newline Sensor  & \underline{aod\_047} &  Mean Aerosol Optical Depth at 470nm.\par & dimension-\newline less & MODIS Terra+Aqua.\\ 
& Satellite \newline Sensor & \underline{uvai} &  Mean UV Aerosol Index. A positive index highlights the presence of UV absorbing aerosol (such as smoke/dust). \par & dimension-\newline less & Sentinel-5P\\ \hline
\pagebreak
\hline
\multirow{3}{4em}{PM10} & Model  & \underline{pm10\_cams} & Mean PM10 concentration at 0m level provided by CAMS  (Ensemble Median - Analysis).\par & $\mu$g/m\textsuperscript{3} & CAMS Model.\\ 
& Ground \newline Sensor  & \underline{pm10\_lcs} &  Mean PM10 concentration ground measurement - Low Cost Sensor ESA monitoring stations.\par & $\mu$g/m\textsuperscript{3} & ESA Air Quality Platform.\\ 
& Ground \newline Sensor & \underline{pm10\_st} &  Mean PM10 concentration ground measurement - ARPA monitoring stations. \par & $\mu$g/m\textsuperscript{3} & ARPA \newline Lombardia\\ \hline

\multirow{3}{4em}{PM2.5} & Model  & \underline{pm25\_cams} & Mean PM2.5 concentration at 0m level provided by CAMS  (Ensemble Median - Analysis).\par &
$\mu$g/m\textsuperscript{3} & CAMS Model.\\ 
& Ground \newline Sensor  & \underline{pm25\_lcs} &  Mean PM2.5 concentration ground measurement - Low Cost Sensor ESA monitoring stations.\par & $\mu$g/m\textsuperscript{3} & ESA Air Quality Platform.\\ 
& Ground \newline Sensor & \underline{pm25\_st} &  Mean PM2.5 concentration ground measurement - ARPA monitoring stations. \par & $\mu$g/m\textsuperscript{3} & ARPA \newline Lombardia\\ \hline
\pagebreak
\hline
\multirow{3}{4em}{SO\textsubscript{2}} & Model  & \underline{so2\_cams} & Mean SO\textsubscript{2} concentration at 0m level provided by CAMS  (Ensemble Median - Analysis).\par & $\mu$g/m\textsuperscript{3} & CAMS Model.\\ 
& Satellite \newline Sensor  & \underline{so2\_s5p} &  Mean SO2  vertical column density at ground level. \par& mol/m\textsuperscript{2} & Sentinel-5P.\\ 
& Ground \newline Sensor & \underline{so2\_st} &  Mean SO\textsubscript{2} concentration ground measurement - ARPA monitoring stations. \par & $\mu$g/m\textsuperscript{3} & ARPA \newline Lombardia.\\ \hline


\multirow{4}{4em}{NO\textsubscript{2}} & Model  & \underline{no2\_cams} & Mean NO\textsubscript{2} concentration at 0m level provided by CAMS  (Ensemble Median - Analysis).\par & $\mu$g/m\textsuperscript{3} & CAMS Model.\\ 
& Satellite \newline Sensor  & \underline{no2\_s5p} &  Mean NO2  vertical column density at ground level.\par & mol/m\textsuperscript{2} & Sentinel-5P.\\ 
& Ground \newline Sensor & \underline{no2\_st} &  Mean NO\textsubscript{2} concentration ground measurement - ARPA monitoring stations. \par & $\mu$g/m\textsuperscript{3} & ARPA \newline Lombardia.\\ 
& Ground \newline Sensor & \underline{no2\_lcs} &  Mean NO\textsubscript{2} concentration ground measurement - Low Cost Sensor ESA monitoring stations. \par & $\mu$g/m\textsuperscript{3} & ESA Air Quality Platform.\\ \hline

\multirow{1}{4em}{NO} & Model  & \underline{no2\_cams} & Mean NO concentration at 0m level provided by CAMS  (Ensemble Median - Analysis). \par& $\mu$g/m\textsuperscript{3} & CAMS Model.\\  \hline

\multirow{1}{4em}{CH\textsubscript{2}O} & Satellite \newline Sensor  & \underline{ch20\_s5p} & Mean Formaldehyde tropospheric column number density. \par& mol/m\textsuperscript{2} & Sentinel-5P.\\  \hline

\multirow{1}{4em}{CH\textsubscript{4}} & Satellite \newline Sensor  & \underline{ch20\_s5p} & Mean column averaged dry air mixing ratio of methane. \par& ppbV & Sentinel-5P.\\  \hline

\multirow{1}{4em}{NO\textsubscript{x}} & Ground \newline Sensor & \underline{nox\_st} &  Mean NO\textsubscript{x} (field: "Ossidi di Azoto") concentration ground measurement - ARPA monitoring stations.\par  & $\mu$g/m\textsuperscript{3} & ARPA \newline Lombardia.\\ \hline

\multirow{1}{4em}{CO\textsubscript{2}} & Ground \newline Sensor & \underline{co2\_lcs} &  Mean CO2 concentration ground measurement - Low Cost Sensor ESA monitoring stations. \par & $\mu$g/m\textsuperscript{3} & ESA Air Quality Platform.\\ \hline
\pagebreak
\hline
\multirow{4}{4em}{CO} & Model  & \underline{co\_cams} & Mean CO concentration at 0m level provided by CAMS  (Ensemble Median - Analysis).\par & $\mu$g/m\textsuperscript{3} & CAMS Model.\\ 
& Satellite \newline Sensor  & \underline{co\_s5p} &  Mean CO vertically integrated column density.\par & mol/m\textsuperscript{2} & Sentinel-5P.\\ 
& Ground \newline Sensor & \underline{co\_st} &  Mean CO concentration ground measurement - ARPA monitoring stations. \par & $\mu$g/m\textsuperscript{3} & ARPA \newline Lombardia.\\ 
& Ground \newline Sensor & \underline{co\_lcs} &  Mean CO concentration ground measurement - Low Cost Sensor ESA monitoring stations. \par & $\mu$g/m\textsuperscript{3} & ESA Air Quality Platform.\\ \hline

\multirow{3}{4em}{O\textsubscript{3}} & Model  & \underline{o3\_cams} & Mean O\textsubscript{3} concentration at 0m level provided by CAMS  (Ensemble Median - Analysis).\par & $\mu$g/m\textsuperscript{3} & CAMS Model.\\ 
& Satellite \newline Sensor  & \underline{03\_s5p} &  Mean O\textsubscript{3} total atmospheric column.\par  & mol/m\textsuperscript{2} & Sentinel-5P.\\ 
& Ground \newline Sensor & \underline{03\_st} &  Mean O\textsubscript{3} concentration ground measurement - ARPA monitoring stations.  \par& $\mu$g/m\textsuperscript{3} & ARPA \newline Lombardia.\\ 
 \hline
 
 \multirow{1}{4em}{CH\textsubscript{2}O}& Satellite \newline Sensor  & \underline{ch20\_s5p} &  Mean Formaldehyde tropospheric column number density. \par & mol/m\textsuperscript{2} & Sentinel-5P.\\ \hline
 
\multirow{1}{4em}{NMVOCs}& Model  & \underline{nmvocs\_cams} & Mean Non-Methane VOCs concentrations at 0m level provided by CAMS.\par & $\mu$g/m\textsuperscript{3} & CAMS Model.\\ \hline

\multirow{3}{4em}{NH\textsubscript{3}} & Model  & \underline{nh3\_cams} & Mean NH\textsubscript{3} concentration at 0m level provided by CAMS  (Ensemble Median - Analysis).\par & $\mu$g/m\textsuperscript{3} & CAMS Model.\\ 
& Satellite \newline Sensor  & \underline{nh3\_lcs} &  Mean NH\textsubscript{3} concentration ground measurement - Low Cost Sensor ESA monitoring stations. \par  & $\mu$g/m\textsuperscript{3} & ESA Air Quality Platform.\\ 
& Ground \newline Sensor & \underline{nh3\_st} &  Mean NH\textsubscript{3} concentration ground measurement - ARPA monitoring stations. \par & $\mu$g/m\textsuperscript{3} & ARPA \newline Lombardia.\\ \hline
\caption{Table of Pollutant variables.}
\end{longtable}
\end{center}
In addition to them, there are also pollutants variables ending with '\_int' (such as 'pm10\_int', 'pm25\_int', 'nh3\_int', etc.). These variables are obtained by interpolating ARPA variables, which are not spatially continued. The explanation of how they are interpolated and how to mitigate the problem of having a limited number of observations from ground sensors along the surface is in one of the next \hyperref[subsec:nan]{subsection}.
\subsubsection{Soil and Vegetation (Table)}

\begin{center}
\setlength{\arrayrulewidth}{1.5pt}
\begin{longtable}{ |p{2.3cm}|p{1.5cm}|p{2.3cm}|p{4cm}|p{1.3cm}|p{2.5cm}| } 
\hline
\textbf{Physical variable} & \textbf{Source type}  & \textbf{Variable name}  & \textbf{Description}  & \textbf{Unit}  & \textbf{Source}\\ 
\hline

\multirow{3}{4em}{Vegetation} & Satellite \newline Sensor  & \underline{siarlX} & Fraction of area in each cell for each agricultural use provided by SIARL Catalog for Lombardy Region.\par & \% & SIARL Lombardia 2019.\\ 
& Satellite \newline Sensor  & \underline{ndvi} &  Mean NDVI cell value over 16 days period.\par & dimen-\newline sionless & USGS Earth Data.\\ 
& Satellite \newline Sensor  & \underline{siarl} &  Majority class for agricultural use provided by SIARL Catalog for Lombardy Region. \par &catego-\newline rical& SIARL Lombardia 2019.\\
\hline
\pagebreak
\hline
\multirow{5}{4em}{Soil} & Model  & \underline{soil\_moist} & Mean volume of water in soil layer 1 (0 - 7 cm) of the ECMWF Integrated Forecasting System. The surface is at 0 cm. The volumetric soil water is associated with the soil texture (or classification), soil depth, and the underlying groundwater level.\par & m\textsuperscript{3}/m\textsuperscript{3} & ERA5 Land Hourly Data.\\ 
& Map Layer  & \underline{soilX} &  Fraction of area for each cell containing the soil type obtained from OpenLandMap soil texture classification.\par & \% & OpenLandMap Soil Texture Class (USDA System).\\ 
& Map Layer  & \underline{soil\_textX} &  Mean NDVI cell value over 16 days period. \par & \% & Basi informative dei suoli - Geoportale Lombardia.\\ 
& Map Layer  & \underline{soil} &  Majority soil type for each pixel from OpenLandMap soil texture classification .\par &catego-\newline rical& OpenLandMap Soil Texture Class (USDA System) .\\ 
& Map Layer  & \underline{soil\_text} &  Majority soil type for each pixel from Carta pedologica 250K (Lombardy Region). \par&catego-\newline rical& Basi informative dei suoli - Geoportale Lombardia.\\ 

\hline
\caption{Table of variables referred to Vegetation and Soil.}

\end{longtable}
\end{center}

\subsubsection{\acrshort{gis} (static layers) (Table)}
\begin{center}
\setlength{\arrayrulewidth}{1.5pt}
\begin{longtable}{ |p{2.2cm}|p{1.5cm}|p{2.3cm}|p{4cm}|p{2.2cm}|p{2.1cm}| } 
\hline
\textbf{Physical variable} & \textbf{Source type}  & \textbf{Variable name}  & \textbf{Description}  & \textbf{Unit}  & \textbf{Source}\\ 
\hline
\multirow{1}{4em}{Geometry} & Map Layer  & \underline{area} & Area of Lombardy Region vector layer in each cell. \par& km\textsuperscript{2} & \acrshort{siarl} Lombardia 2019.\\ 
\hline
\multirow{1}{4em}{Calendar} & Map Layer  & \underline{cal} & JSON file containing the time-ranges used for data processing. \par& day & -\\ 
\hline

\multirow{1}{4em}{Population} & Map Layer  & \underline{pop} & Population for each cell. & dimensionless& Gridded Population of the World (GPW).\\ \hline

\multirow{2}{4em}{Land use and cover} & Map Layer  & \underline{dsfX} & Land use fraction for each cell containing the classification provided by DUSAF Catalog (Lombardy Region). & \% (fraction for each cell) & DUSAF Lombardia 2018.\\ \hline
Map Layer  & \underline{dusaf} & Cover & Land Use majority class for each cell provided by DUSAF Catalog (Lombardy Region). &categorical & \acrshort{dusaf} Lombardia 2018.\\
\hline
\pagebreak
\hline
\multirow{3}{4em}{Terrain} & Map Layer  & \underline{h\_mean} & DTM average elevation for each pixel. & m & Geoportale Lombardia 2019.\\ 
& Map Layer  & \underline{aspect\_major} & Aspect derived from DTM. Majority pixel aspect. &  °N & Geoportale Lombardia 2019.\\ 
& Map Layer  & \underline{slope\_mean} & Average slope derived from DTM. &  °N& Geoportale Lombardia 2019.\\ 
\hline
\multirow{6}{4em}{Road Infrastru-\newline ctures} & Map Layer  & \underline{int\_prim} & Density of intersection nodes between primary roads for each cell (including highways). & intersections/\newline km\textsuperscript{2} & Geoportale Lombardia 2019.\\ 
& Map Layer  & \underline{int\_prim\_sec} & Density of intersection nodes between primary and secondary roads for each cell. & intersections/\newline km\textsuperscript{2} & Geoportale Lombardia 2019.\\ 
& Map Layer  & \underline{int\_sec} & Density of intersection nodes between secondary roads for each cell. & intersections/\newline km\textsuperscript{2} & Geoportale Lombardia 2019.\\ 
& Map Layer  & \underline{prim\_road} & Density of primary importance roads for Lombardy Region inside for each. & km/km\textsuperscript{2} & Geoportale Lombardia 2019.\\ 
& Map Layer  & \underline{sec\_road} & Density of secondary importance roads for the Lombardy region for each cell. & km/km\textsuperscript{2} & Geoportale Lombardia 2019.\\ 
& Map Layer  & \underline{highway} & Density of highways for Lombardy Region inside for cell divided. & km/km\textsuperscript{2} & Geoportale Lombardia 2019.\\ 
\hline

\multirow{1}{4em}{Farms building} & Map Layer  & \underline{farms} & Fraction of area covered by farms inside the cell. Obtained from DUSAF dataset. & \% (fraction for each cell) & DUSAF Lombardia 2018.\\ \hline
\multirow{1}{4em}{Breeding Farms} & Map Layer  & \underline{farm\_type} & Density of farms classified by breed type for each cell: poultry, pigs, sheeps. & \#farms/km\textsuperscript{2} & DUSAF Lombardia 2018.\\ \hline
\multirow{1}{4em}{Air quality zones} & Map Layer  & \underline{aq\_zone} & Majority class of a given air quality zone in each cell. &categorical & Geoportale Lombardia.\\ \hline
\multirow{1}{4em}{Climate zones} & Map Layer  & \underline{clim\_zone} & Majority class of a given air quality zone in each cell. &categorical& - \\ \hline


\hline
\caption{Table of Static GIS variables.}

\end{longtable}
\end{center}
\pagebreak

\subsubsection{Categorical Variables (table)}
Categorical data are identified with names or labels given to them as value. Even if are represented by numbers, they don't have the same mathematical meaning as a numerical value. 
This type of data is discarded during the pre-processing phase, since feature selection is done exclusively on numerical input and output values. \\
In the following table is explained the semantics of the values assumed.
\bigbreak
\begin{center} 
\setlength{\arrayrulewidth}{1.5pt}
\begin{longtable}{ |p{2.5cm}|p{10cm}| } 
\hline
\textbf{Variable name} & \textbf{Note}\\ 
\hline

 \multicolumn{2}{|c|}{\textbf{Meteo}} \\
\hline
 \underline{wind\_dir\_st}  \newline \newline (Wind direction from ground sensor divided in 8 sectors). & 1 = North: 0° - 22.5° / 337.5° - 360°; \newline2 = North-East: 22.5° - 67.5°; \newline3 = East: 67.5° - 112.5°;  \newline4 = South-East: 112.5° - 157.5°; \newline5 = South: 157.5° - 202.5°; \newline6 = South-West: 202.5° - 247.5°; \newline7 = West: 247.5° - 292.5°; 8 = North-West: 292.5° - 337.5°\\ \hline
 \multicolumn{2}{|c|}{\textbf{Soil and Vegetation}} \\ \hline
 \underline{siarl} \newline \newline (Majority class for agricultural use provided by SIARL Catalog for Lombardy Region). & 2 = Cereal; \newline9 = Mais; \newline12 = Rice;\\  \hline
 \underline{soil} \newline \newline (Majority soil type for each pixel from OpenLandMap soil texture classification). &  1 = Clay; \newline2 = Silty Clay; \newline3 = Sandy Clay; \newline4 = Clay Loeam; \newline5 = Silty Clay Loam; \newline6 = Sandy Clay Loam; \newline7 = Loam;\newline8 = Silt Loam; \newline9 = Sandy Loam; \newline10 = Silt; \newline11 = Loamy Sand; \newline12 = Sand;  \\ \hline 
\underline{soil\_text} \newline \newline (Majority soil type for each pixel from Carta pedologica 250K). & 1 = Fine clay;\newline 2 = Very fine clay;\newline  3 = Fine loose;\newline  4 = Coarse loose;\newline  5 = Fine silty;\newline  6 = Coarse silty;\newline  7 = Skeletal-clayey sand;\newline  9 = Skeletal-loose;\newline  10 = skeletal-sand; \\ \hline
 \multicolumn{2}{|c|}{\textbf{GIS (Static layers)}} \\ \hline
 \underline{dusaf} \newline \newline (Land use and cover). & 2 = Agricultural areas; \newline3 = Wooded territories and semi-natural environments; \newline4 = Wetlands; \newline5 = Water bodies; \newline11 = Urbanised areas; \newline12 = Production facilities, large plants and communication networks; \newline13 = Mining areas, landfills, construction sites, waste and abandoned land; \newline14 = Non-agricultural green areas; \\
\hline
 \underline{aq\_zone}\newline \newline (Air quality zones) & 1 = Highly urbanized plains; \newline 2 = Plains; \newline 3 = Prealpi, Appennino and mountains;\newline 4 = Valley floor Agg; 
\newline5 = Urban agglomarated area (Milano, Bergamo, Brescia);\\
\hline
 \underline{clim\_zone}\newline \newline (Climate zones) & 1 = Alpi;\newline 2 = Prealpi Occidentali; \newline 3 = Prealpi Orientali;\newline 4 = Pianura Occidentale;\newline 5 =  Pianura Centrale;\newline 6 = Pianura Orientale; 
 \\
\hline
\caption{Table of categorical variables with their values legend.}



\end{longtable}
\end{center}

In order to examine the behaviour of each variable in a data set over time, several grid data are collected, each one with a different resolution, period and configuration.
\subsection{Spatial resolution}
Vector grids that are used in the D-DUST project are two and are generated by the spatial resolution of the source provider. 

\begin{itemize}
\item Grids with Copernicus \acrshort{cams} resolution (0.1°); resolution with Copernicus CAMS (European);
\item Grids with 0.01° Grid defined with maximum one \acrshort{arpa} station for each cell;
\end{itemize}

Data are scaled and fit in each spatial resolution grid in order to better analyze the final output model by considering each of them. 

\par
\subsection{Period}  
Data collected are dated 2021 because they are the most recent and, with reference to 2020, those that are not particularly affected by emission reduction caused by lockdown for the COVID-19 pandemic \cite{bontempi2022analysis}. 
\pagebreak
In this case study, grid data were chosen by considering the effect of intensive farming activities , with these particular conditions:
    \begin{itemize}
        \item To have the right conditions for farming, in the period chosen the terrain shouldn't be frozen (Temperature > 0°C). So I selected data coming from the spring, summer and autumn periods (discarding winter);
        \item For better highlighting the effect of intensive farming activities with the usage of fertilizer and pesticides, which are the main pollution emission factors, weeks with no precipitations were selected;
\end{itemize}
In this way, several grid data were chosen from 5 different weeks over the year:
\begin{itemize}
    \item 24 March - 31 March 2021;
    \item 18 April - 25 April 2021;
    \item 17 July - 24 July 2021;
    \item 3 September - 10 September 2021;
    \item 7 October - 14 October 2021
\end{itemize}

\subsection{Target Variables}
In this case study, the chosen target variables represent the pollution phenomena correlated to intensive farming activities , such as PM25 and ammonia emissions. \\
One of the objectives of this step is to detect the main pollutant factors that contribute further to the training of PM25 or NH3 emissions.\\
PM25 and ammonia (NH3) provided by ARPA sensors are the ones chosen as target variables('pm25\_st' and 'nh3\_st').\\
I chose these measurements because are significant primary pollutants from intensive farming \cite{aneja2008farming}. Moreover, air quality monitoring is traditionally measured by fixed ground-sensor.
\\
ARPA air quality monitoring stations, which are operating 24 hours a day 365 days a year, are periodically checked and subject to maintenance, to ensure proper functioning and reliability.\par
Data covariates are divided between input (X) and output variables (Y). X represents all the variables collected in the previous part, except for the pollutant to be analyzed and modelled (such as PM25 or Ammonia), which is assigned to the Y variable.
\subsection{Mountains}
Another important parameter configuration used is to filter or not the cell covered by mountains or not (climate zone = 1/2/3 of Alpes and Prealpes). \\ 
In this way, I run an extra test for each resolution in which I include only the grid cells that belongs to urban and land areas.
I choose also this different configuration since a focus of the D-DUST project is on the Po Valley, an important area related to agriculture.
Since I assume 2 different resolutions and with the possibility to include or not the mountain zones, I obtained 4 different configurations: 
\par
\begin{table}[H]
    \centering
    \begin{tabular}{|l|}
    \hline
        10km resolution (0.1°) with all clim\_zone (zone pedoclimatiche)  \\ \hline
        10km resolution (0.1°) without Alpes and Prealpes (clim\_zone > 3) \\ \hline
        1km resolution (0.01°) with all clim\_zone (zone pedoclimatiche)   \\ \hline
        1km resolution (0.01°) without Alpes and Prealpes (clim\_zone > 3)  \\ \hline
    \end{tabular}
\end{table}
\section{Data Cleaning of NaN and low-variance values }
\label{subsec:nan}
In my work, I considered as target variable the PM25 and Ammonia measured by ARPA ground sensors.\\
\begin{comment}
Air quality monitoring is usually carried out through ground sensors networks, which represent the primary air quality data source by governance.
\end{comment}
\newline
In the processed grids there is the problem that a given value provided by measurement tools could be NaN. \\
This is caused by the fact that each ground sensor observation, when it is collected in a grid, is assigned to a cell by its location. \\
For this, many cells are not assigned by any ground sensor observations during this step.
So it turns out that variables provided by ARPA and \acrshort{esa} ground sensors (with the label that ends with '\_st' and '\_lcs' respectively) have many NaN cells in the grids in each different resolution (in the figure \ref{fig:knn-interpolated} the only cells containing the values provided by the ground sensors are expressed with the red cross).
\begin{figure}[H] 
\centering
\includegraphics[scale=0.4]{images/cell_with_sensors.png}
  \caption{Graphical representation of the cells which have ground sensor masuraments in the grid at 0.1° at resolution.}
 \label{fig:knn-interpolated}
\end{figure}
However, no country in the world has yet established a monitoring network with a fully satisfying coverage \cite{liu2018improve}. Even the United States (US), which is characterized by a relatively developed PM2.5 ground monitoring network with 2500 stations has many areas unmonitored \cite{liu2018improve}. \par
Due to the fact that with the use of data cleaning it results a data set with a number of samples too limited to be used fot FS or any models.
By seeing figure \ref{fig:cells}, there are respectevely only 34 and 8 observations if we considered PM2.5 and Ammonia as target variable.
In order to mitigate this, I adopted an approach in order to increase its size with interpolation.
This approach was not accurately justified because is used only to increase the size of cells with ground sensor measuraments for making applicable the FS procedure.
I previously performed a classifier k nearest neighbor  \cite{taunk2019brief} to detect the buffer of cells close to the location of the ground station measurement.\\
To increase the number of observations provided by the limited ARPA stations in Lombardy, a k-nearest neighbors algorithm is applied to add the buffer of values (with k, respectively, equal to 10 for 0.1° and 30 for 0.01° resolution). \\ 

\begin{figure}
\begin{multicols}{2}
\begin{verbatim}
****************************
target_variable: 'pm25_st'
resolution = 0.1° (10 km)
old size: 34
new size: 173 
****************************
target_variable: 'pm25_st'
resolution = 0.01° (1 km)
old size: 34
new size: 963 
****************************
****************************
target_variable: 'nh3_st'
resolution = 0.1° (10 km)
old size: 8
new size: 69 
****************************
target_variable: 'nh3_st'
resolution = 0.01° (1 km)
old size: 8
new size: 240 
****************************
\end{verbatim}
\end{multicols}
\caption{In this figure the results of the increment of the number of cells is shown for each different target variable and resolution.}
\label{fig:cells}
\end{figure}

\begin{figure}[H] 
    \centering
    \subfloat[10 km resolution including mountains.]{%
        \includegraphics[width=0.55\textwidth]{images/0_1_knn.png}%
        %
        }%
    \hfill%
    \subfloat[1 km resolution including mountains.]{%
        \includegraphics[width=0.55\textwidth]{images/0_01_knn.png}%
        %
        }%
    \caption{Performance in terms of MAE, RMSE, MSE and R\textsuperscript{2} of the Random Forest model with PM2.5 as target variable in function of the K chosen by \acrshort{knn} applied in the data cleaning.}
    \label{fig:knn_chosen}
\end{figure}
Then, these cells are set with values computed using a \gls{rbf} \cite{wright2003radial} which are separately stored in variables with the name that ends with '\_int'. RBF was performed using a linear interpolation.\\
The given procedure and validation results are available on GitHub\footnote{\url{https://github.com/gisgeolab/D-DUST/blob/WP2/Ground\%20Sensor\%20Variables\%20Request\%20.ipynb}}.
The reason of the choice of k is to find a value which could have been a trade-off between dimension of k and both the accuracy and processing time that allows us to apply FS.
Accuracy could be seen in the figure \ref{fig:knn_chosen} where the error metrics of the Random Forest model (described in the next section \ref{sec:modelling}) are shown in function of the k choosen.
Instead, in the figure ssss is shown the processing time taken to run this procedure for each different resolution in function of the k choosen. 
TODO
\par
Before the feature selection phase I remove variable with low variance.
A feature with low variance is characterized by having observations very close to the mean value and thus almost constant.
Features with constant values should be discarded because of its low variability.
In this study \cite{kuhn2008building}, this was performed with a method called  nearZeroVar from \gls{caret}, a package in R.
As suggested by this study, I discarded variables which has the following parameters: 
\begin{itemize}
    \item the percentage of unique values greater than 20\%;
    \item the ratio of the most prevalent over the second most prevalent value greater than 20;
\end{itemize} 
This choice in the article is made because of the presence of a very frequent value. In this case study instead, there are many variables related to the fractional land use (such as siarlX and dsfX) which assign 0 to most of the cells. TODO

\pagebreak
\section{Feature Selection}
In this phase, I got the list of the most significant variables for each different configuration.\\
For doing that, after the features are selected for each period, an average of them is performed using again Borda Count algorithm. What is obtained is a general set of variables for each different configuration.\\
This was computed by averaging the feature selected for each different period with Borda Count another time (figure \ref{fig:overview_configuration}).\\
\bigbreak
\begin{figure}[H]
    \centering
    \includegraphics[width=.9\textwidth]{images/overview_results_configuration.png}
    \caption{The following diagram aims to visualize how the results for each different configuration are provided. First of all,  the final FS results for each period are computed. Then, these are averaged through the Borda Count algorithm again.}
    \label{fig:overview_configuration}
\end{figure}


In the \ref{fig:test_params} is shown a summary of the parameter of each test execution.
It can be observed that for each parameter there are different configurations: there are 2 resolutions, 2 climate zones parameters and 5 periods. Therefore, for each different target variable, 2x2x5 = 20 different tests were run for both ammonia and fine particulate (40 in total). \\
The results of different periods are grouped and averaged for simplicity. In total in this work are attached 8 different results. Each result will be in detail pointed out in the final Appendix \ref{chap:appendix}.
\begin{figure}[H]
    \centering
    \includegraphics[width=.9\textwidth]{images/test_param.png}
    \caption{Parameter taken in consideration for tests execution.}
    \label{fig:test_params}
\end{figure}
\bigbreak
\section{Data Modelling}
\label{sec:modelling}
In this phase, I evaluate models trained by the variable selected by \acrshort{fs} in order to analyze their interpretability and performance.\\
In detail, the tests run are built and validated using a k-fold cross-validation (k=5) which is one of the most used procedure for evaluation of model performance.
I performed a 5-fold since brings better performances from some studies in litherature  \cite{anguita2005k} \cite{jung2020evaluation}.
I perform tests in 2 different ways, explained in the next subsections.
\subsection{Test for each period with CAMS comparison}
The validation made on the model was performed with 2 different variable values. They have been chosen:
\begin{itemize}
\item values of the target variable (ARPA measurements);
\item values assumed by the CAMS model;
\end{itemize}
Such a procedure was applied both to evaluate the performance of the current model and to compare it to the CAMS model.\\
In this way, a comparison between results from these 2 different validations is performed, aiming to detect how much the models perform better in this local scale.\\
This is was applied for each data set obtained by the combinations of 4 configurations, 5 periods and 2 target variables (4x5x2 = 40 different model testes). 
\subsection{Test for analysing FS impact on performance}
In order to evaluate how Feature Selection effectively improves the accuracy of the model, a set of tests are run and configured differently from the previous.\\ 
Data from different periods were merged in order to develop the model through a temporal hold-out validation, where the data are used in this way:
\begin{itemize}
    \item grid data from March, April and July, September periods were used from training;
    \item grid data from October period was used for validation;
\end{itemize}
In this way, model performance should be evaluated by means of external data, which comes from another period.\\
However, before running this, I have to deal with the collinearity problem, which can cause bias in the accuracy estimation.\\ In order to mitigate this, I perform a feature cluster, grouping collinear variables between them. I consider a variable to be collinear to another if and only if the Pearson correlation index is greater than or equal to 0.8, as a study suggests \cite{shrestha2020detecting}. \\
After that, I transform each cluster of variables using the first \gls{pca} component (called 'PC\_0' component below). In this way, I solve collinearity in the independent variables, without particularly losing the features interpretability.\\
Behind them are attached the groups of collinear variables and the new covariates used in the Feature Selection for PM2.5 as the target variable, with 10km and 1km resolutions and including mountain zones.
I group and illustrate collinear variables in clusters with the help of a script available at this link\footnote{\url{https://www.kaggle.com/code/nassehkhodaie/fixing-multicollinearity-by-feature-clustering/data?select=collinearity_finder_treater_py.py}}.
The group of variables averaged with PCA is shown in the next page.

\pagebreak
\subsubsection{Resolution = 10 km}
\begin{verbatim}
**********
cluster_0
feature name = 'dsf2', 'dsf3', 'press'
explained variance by PC_0 = 0.89
**********
cluster_1
feature name = 'dsf14', 'dsf11', 'pop', 'sec_road', 'int_sec', 'dsf12'
explained variance by PC_0 = 0.83
**********
cluster_2
feature name = 'prim_road', 'int_prim_sec'
explained variance by PC_0 = 0.90
**********
cluster_3
feature name = 'nh3_cams', 'siarl9', 'farms'
explained variance by PC_0 = 0.88
**********
cluster_4
feature name = 'o3_s5p', 'uvai'
explained variance by PC_0 = 0.94
**********
cluster_5
feature name = 'nmvocs_cams', 'pm10_int', 'pm10_cams', 'no2_cams', 'no_cams', 'pm25_cams', 'co_cams'
explained variance by PC_0 = 0.75
**********
cluster_6
feature name = 'no2_int', 'nox_int'
explained variance by PC_0 = 0.98
**********
cluster_7
feature name = 'o3_cams', 'o3_int'
explained variance by PC_0 = 0.97
**********
cluster_8
feature name = 'temp_int', 'temp_2m'
explained variance by PC_0 = 0.98

New features: ['dsf5', 'dsf13', 'soil7', 'soil_text3', 'soil_text4', 'farm_pigs',
       'farm_poultry', 'farm_sheep', 'ndvi', 'prec', 'n_wind', 'e_wind',
       'soil_moist', 'aod_047', 'co_s5p', 'dust_cams', 'so2_cams', 'co_int',
       'nh3_int', 'so2_int', 'air_hum_int', 'prec_int', 'rad_glob_int',
       'wind_speed_int', 'cluster_0_pc0', 'cluster_1_pc0', 'cluster_2_pc0',
       'cluster_3_pc0', 'cluster_4_pc0', 'cluster_5_pc0', 'cluster_6_pc0',
       'cluster_7_pc0', 'cluster_8_pc0']
\end{verbatim}
\pagebreak
\subsubsection{Resolution = 1 km}
\begin{verbatim}
**********
cluster_0
feature name = 'h_mean', 'slope_mean'
number of PC needed = 1
explained variance by PC_0 = 0.9343247371259237
**********
cluster_1
feature name = 'uvai', 'o3_s5p'
number of PC needed = 1
explained variance by PC_0 = 0.9502248817971359
**********
cluster_2
feature name = 'nmvocs_cams', 'no_cams', 'no2_cams', 'pm10_int', 'co_cams', 'pm10_cams', 'pm25_cams'
number of PC needed = 1
explained variance by PC_0 = 0.7356224143251505
**********
cluster_3
feature name = 'no2_int', 'nox_int'
number of PC needed = 1
explained variance by PC_0 = 0.9760779670033077
**********
cluster_4
feature name = 'o3_cams', 'o3_int'
number of PC needed = 1
explained variance by PC_0 = 0.9678067765212931
**********
cluster_5
feature name = 'temp_int', 'temp_2m'
number of PC needed = 1
explained variance by PC_0 = 0.9726802610839925
New features:  ['aspect_major', 'pop', 'dsf2', 'dsf11', 'dsf12', 'soil_text3',
       'soil_text4', 'soil_text9', 'sec_road', 'ndvi', 'prec', 'press',
       'n_wind', 'e_wind', 'soil_moist', 'aod_047', 'co_s5p', 'nh3_cams',
       'dust_cams', 'so2_cams', 'co_int', 'nh3_int', 'so2_int', 'air_hum_int',
       'rad_glob_int', 'wind_speed_int', 'cluster_0_pc0', 'cluster_1_pc0',
       'cluster_2_pc0', 'cluster_3_pc0', 'cluster_4_pc0', 'cluster_5_pc0']
\end{verbatim}


\chapter{Results and Interpretations }
 \label{chap:res}
In this chapter are collected interpretations of the results obtained in the Feature Selection and the models' performance. All other tests not included in this chapter are available in the appendix \ref{chap:appendix}.
\\
\section{Feature Selection Results}
In this section the results obtained for each target variable selected are explained in relation to the:
\begin{itemize}
    \item votes received through Borda Count algorithm;
    \item positive or negative correlation assumed in the filter methods;
    \item confirmation in literature where a given correlation occurred; 
\end{itemize}
Similarities between the tests run for each target variable are visible for many factors that influence both particulate matter and ammonia.\\
This can be observed by the similarity of both positive and negative correlation from indexes from filter methods. \\
In the figure \ref{fig:pearson_general} are shown the results of the Pearson index, a filter method used during the FS. 
\begin{figure}[H]
\centering
\subfloat[pm25\_st as target variable.]{\includegraphics[scale =0.45]{images/tests/pm25pearson_october_0_01_mountains.png}}\\
\subfloat[nh3\_st as target variable.]{\includegraphics[scale =0.45]{images/tests/nh3_pearson_october_0_01_mountains.png}}
\caption{Pearson index results obtained in the period between 7-14 October with 1km resolution including mountains.}
\label{fig:pearson_general}
\end{figure}
The two bar plots attached aims to compare the scores obtained in the same period by the 2 target variables chosen in my tests in the October period.
One similarity between them is the influence of meteorological and morphological parameters.
By figure it appears that DTM average elevation and slope ('h\_mean' and 'slope\_mean' respectively) are correlated in the same way in the test of both target variables, since air pollution affects more urban than mountain areas.\\
Pollutants are negatively correlated to wind speed ('e\_wind'), since their concentration tends to turn down with it.\\ 
Indeed, the wind is a marker for the horizontal transport of air pollutants. Pressure also contributes to them ('press'), since it causes stable atmospheric conditions that make pollutants harder to be disseminated.
\par
In the test run the findings didn't  consistent differences if Alpes and Prealpes zones are included or not.\\
The only slight diversity found is about the votes obtained by pollutant and vegetation variables, which tend to be generally higher if mountain zones are excluded.
\subsection{PM2.5 as target variable}
Fine particulate matter (PM2.5) is one of the most common air pollutants in our environment. 
It comes primarily from transport vehicles (such as cars, trucks and buses), industry, burning of fuels and household activities. \\
During the 5 periods, it has been shown that PM2.5, as the other pollutants, changes over time. Air pollution changes accordingly to the type of climate and atmospheric conditions.\\
\begin{table}[H]
\centering
\begin{tabular}{lrrrrr}
\toprule
 Statistic &  24/03-31/03 &  18/04-25/04 &  17/07-24/07 &  3/09-10/09 &  7/10-14/10 \\
\midrule
  Mean  &        30.217  &         18.203  &         12.974 &       15.450 &       14.878 \\
Median  &        27.938 &        17.938 &        13.063 &       14.75 &       15.281 \\
 Standard deviation &        3.772 &        3.462 &        1.831 &       4.000 &       3.166 \\
  Min value &        20.0  &        11.875 &        9.5 &        8.625 &       9.0 \\
  Max value &        42.0 &        25.714 &        17.375 &        27.875 &       19.625 \\
\bottomrule
\end{tabular}
\caption{Summmary statistics (measured in $\mu$g/m\textsuperscript{3}) of the PM2.5 provided by ARPA with 10 km resolution.}
\label{tab:statspm25}
\end{table}
\begin{figure}[H]
    \centering
    \includegraphics[scale=0.8]{images/pm25_values.png}
    \caption{Graphical representation of PM2.5 variation over the 5 periods, in relation of its mean, median and standard deviation. Each measurament are expressed with $\mu$g/m\textsuperscript{3} as unite of measure.}
    \label{fig:graphstatspm2.5}
\end{figure}
Air pollution analysis can confirm the presence of a high amount in the winter and heating season and a low quantity in the summer months \cite{cichowicz2017dispersion}. This is not applied to the ozone ('o3\_int' and 'o3\_cams'), which should be, on the contrary, higher during summer due to a combination of heat and sunlight which reacts with nitrogen oxides (NOx) and volatile organic compounds.\\
Indeed, ozone receives in the results many votes for its negative correlation (the highest in summer period), because is inversely related to PM2.5.\\
By the bar plots attached in the figure \ref{fig:fs_pm25}, it is clear that most of the pollutants have a strong or moderate correlation with PM2.5. \\
It can be shown in figure \ref{fig:pearson_general} that the entire set of pollutants variables is almost all positively correlated to it.\par
In particular, PM10 provided by ARPA and CAMS, in the various test executed is one of the most correlated pollutants. Indeed, 'pm10\_int' and 'pm10\_cams' have always a high number of votes. 
A strong positive correlation between PM2.5 and PM10 is also demonstrated in the literature \cite{zhou2016concentrations}.
Another variable which received a strong number of votes is the PM2.5 modelled by CAMS ('pm25\_cams'). \\
Both variables, in fact, measure the same pollution phenomena as the target variables. This finding makes us understand how both describe the same pollution phenomena, moving in the same direction. Indeed, the correlation coefficient between them is close to 1 in all tests executed.\\
It's visible a correlation with nitrogen dioxide('no2\_int'), a strong greenhouse gas which is related, as particulate matter, to biomass burning, industrial processes, households and road transport \cite{zellner2000john} \cite{maranzano2022air}.\\
From nitrogen dioxide derives also the chemical relation with the other nitrogen oxides ('nox\_int'), such as nitrogen monoxide ('no\_int') and carbon monoxide ('co\_int', 'co\_cams, 'co\_s5p'). This could be explained by the fact that they contribute as a secondary pollutant to PM2.5 formation \cite{xie2015spatiotemporal}.
\par
Observing the results regarding intensive farming activities , it's evident that ammonia ('nh3\_int' and 'nh3\_cams') received, as PM10, most votes of all in the spring period due to its strong positive correlation. \\
This happens because NH3 can react in the atmosphere to form ammonium salts in the presence of acid species \cite{viatte2021ammonia}.\\
We can assume that the main source of ammonia emission is intensive farming activities  since it was responsible for 92\% of the total by EEA country members in 2017 \cite{maranzano2022air}.\\
Indeed, during these periods, the application of fertilizer contributes to ammonia and particulate matter.\\
Significantly more fertilizer is applied in the spring than in the summer and winter seasons due to crop cycles \cite{goebes2003ammonia}.\\
Therefore, we can assume that intensive farming activities  is one of the factors that influence the formation of PM2.5, with the greatest contribution during a certain period of the year.
\begin{figure}[H]
\centering
\subfloat[Borda Count algorithm results.]{\includegraphics[scale =0.45]{images/tests/0_01_mountainspm25_st.png}\label{fig:a}}\\
\subfloat[Pearson index results in the  24-31 March and 18-25 April periods.]{\includegraphics[scale =0.40]{images/tests/ammonia_effect_on_pm25_001mountains.png}\label{fig:b}}
\caption{FS results assuming PM2.5 as target variable with 10 Km resolution including mountains. In figure \ref{fig:a} it is possible to observe the features ordered in decreasing order concerning the votes received by Borda-Count algorithm. Figure \ref{fig:b} highlights the positive correlation assumed by ammonia variables in the spring period. }
\label{fig:fs_pm25}
\end{figure}
\subsection{NH3 as target variable}
Ammonia (NH3) is a reactive and soluble alkaline gas. It is one of the main sources of nitrogen pollution and comes from natural and anthropogenic sources, such as agriculture.\\
The process of ammonia evaporation commonly takes place when nitrogen is originated by the urea of animal livestock, and fertilize. \\
In the results obtained along the five periods, the ammonia provided by the CAMS model ('nh3\_cams') stands out, because it is strongly correlated, and represents the same pollutant as well.
We can observe in figure \ref{fig:fs_nh3} also variable related to PM10 and PM2.5 ('pm10\_int', 'pm10\_cams', 'pm25\_int' and 'pm25\_cams'), because ammonia contribute for the formation of secondary particulate matter \cite{dai2019concentrations} \cite{zhu2015sources}.\\
Correlation of ammonia with respect to intensive agriculture should be feasible by the high positive correlation with agricultural areas (modelled by 'dsf2' variable), in particular used for maize cultivation ('siarl9') .\\
It's observable also the votes received by the variable that models the soil moisture, always with a positive correlation with respect to the target variable. This could be explained by the abundance of ammonia-oxidizing bacteria that increase with soil moisture \cite{avrahami2007response}.  \\
Other important weighted features are the ones related to intense farming ('farms' and 'farm\_pigs') which are responsible for ammonia release thanks to the chemical reaction of urea.\\
In fact, animal urine and faeces imply the release of ammonia and methane in the atmosphere, respectively \cite{saggar2004review}.

So we can suggest that ammonia in Lombardy should be very related to the use of fertilizer in agricultural areas and animal livestock in farms.
\bigbreak
\pagebreak
\clearpage
\begin{figure}[H]
\centering
\includegraphics[scale =0.50]{images/tests/0_1_nomountainsnh3_st.png}
\caption{FS results obtained with Ammonia (NH3) as target variable with 1 Km resolution excluding mountains.}
\label{fig:fs_nh3}
\end{figure}
So overall, the results obtained in feature selection demonstrate two things.\\
First, fine particulate should be very correlated with ammonia in the manuring period (spring). Second, ammonia is strictly related to agriculture and farming activity, in accordance with previous studies.\pagebreak
\section{Data Modelling Results}
This section collected the interpretation of the ML models results, paying attention not only to the results achieved by the ARPA and CAMS models but also to how feature selection could improve performance results. 
\subsection{Interpretation of validation results with ARPA sensor}
Models assure consistent accuracy.
The errors obtained in the results by the validation through particle matter decrease as a function of the higher resolution.\\
Higher resolution implies a larger number of samples for training and so a better accuracy as consequence.\\
Instead, in the models for ammonia estimation, there are no relevant differences between the 2 resolutions.
With PM2.5 as the target variable at 10 km resolution, there is a lower statistical accuracy than at 1km, with MAE respectively around 1 and 0.2 ug/m\textsuperscript{3}(Tables \ref{tab:res1km}  and \ref{tab:res1km}). 
\begin{table}[H]
\begin{tabular}{lrrrrr}
\toprule
  &  24/03-31/03 &  18/04-25/04 &  17/07-24/07 &  3/09-10/09 &  7/10-14/10 \\
\midrule
  MAE\_sensor &        1.448 &        1.093 &        0.646 &       0.898 &       0.841 \\
RMSE\_sensor &        1.940 &        1.362 &        0.868 &       1.165 &       1.080 \\
 MSE\_sensor &        3.772 &        1.874 &        0.776 &       1.376 &       1.197 \\
  R2\_sensor &        0.872 &        0.766 &        0.683 &       0.870 &       0.793 \\
\bottomrule
\end{tabular}
\caption{Random Forest prediction for PM2.5 at 10 km, excluding zones with mountains.}
\label{tab:res10km}
\end{table}
\begin{table}[H]
\begin{tabular}{lrrrrr}
\toprule
  &  24/03-31/03 &  18/04-25/04 &  17/07-24/07 &  3/09-10/09 &  7/10-14/10 \\
\midrule
  MAE\_sensor &        0.251 &        0.186 &        0.147 &       0.172 &       0.132 \\
RMSE\_sensor &        0.503 &        0.361 &        0.255 &       0.285 &       0.240 \\
 MSE\_sensor &        0.283 &        0.138 &        0.066 &       0.084 &       0.059 \\
  R2\_sensor &        0.992 &        0.985 &        0.981 &       0.994 &       0.993 \\
\bottomrule
\end{tabular}
\caption{Random Forest prediction for PM2.5 at 1 km, including zones with mountains.}
\label{tab:res1km}
\end{table}

\begin{table}[H]
\begin{tabular}{lrrrrr}
\toprule
  &  24/03-31/03 &  18/04-25/04 &  17/07-24/07 &  3/09-10/09 &  7/10-14/10 \\
\midrule
 MAE\_sensor &        0.483 &        0.280 &        0.526 &       0.939 &       0.386 \\
RMSE\_sensor &        0.993 &        0.558 &        1.184 &       2.316 &       0.860 \\
 MSE\_sensor &        1.199 &        0.430 &        1.556 &       6.147 &       1.121 \\
  R2\_sensor &        0.997 &        0.992 &        0.994 &       0.987 &       0.990 \\
\bottomrule
\end{tabular}
\caption{Random Forest prediction for NH3 at 1 km, excluding zones with mountains.}
\label{tab:nh3RF}
\end{table}
\begin{table}[H]
\begin{tabular}{lrrrrr}
\toprule
  &  24/03-31/03 &  18/04-25/04 &  17/07-24/07 &  3/09-10/09 &  7/10-14/10 \\
\midrule
 MAE\_sensor &        1.904 &        2.287 &        3.126 &       2.743 &       3.146 \\
RMSE\_sensor &        2.570 &        3.341 &        4.205 &       3.841 &       4.465 \\
 MSE\_sensor &        6.867 &       11.579 &       18.068 &      16.405 &      21.579 \\
  R2\_sensor &        0.984 &        0.800 &        0.927 &       0.965 &       0.852 \\
\bottomrule
\end{tabular}
\caption{Neural Network prediction for NH3 at 1 km, excluding zones with mountains.}
\label{tab:nh3NN}
\end{table}
In the model for the estimation of ammonia, instead, this error is clearly higher (Tables \ref{tab:nh3RF} and \ref{tab:nh3NN}).
Nevertheless, the results are substantially accurate as well. \\
This difference should be given by a different number of ground sensors for each pollutant. The sample for the measurement of particulate matter is larger than the one for ammonia (Figure \ref{fig:comparison-sensors}).
\begin{figure}[H] 
    \centering
    \subfloat[PM2.5 as target variable.]{%
        \includegraphics[width=0.5\textwidth]{images/pm25_sensors.png}%
        %
        }%
    \hfill%
    \subfloat[NH3 as target variable.]{%
        \includegraphics[width=0.5\textwidth]{images/nh3_sensors.png}%
        %
        }%
    \caption{In these images are shown the observations at 10km resolution of each different target variable used in this case study. \\
    The legend shows whether the cell has an interpolated value with KNN or not. In addition, ground sensors from which interpolation is performed are added.\\ 
    The sample number passed from 28 to 173 for PM2.5 as target variable and from 8 to 69 for NH3.}
    \label{fig:comparison-sensors}
\end{figure}
We can observe that generally, the Random Forest model makes more accurate predictions than the Neural Network by Keras (Tables \ref{tab:nh3RF} and \ref{tab:nh3NN}). \\
This may be since RF algorithm uses the ensemble learning method for regression, a technique that joins estimation from multiple models to have more accurate predictions than a single model.\\ 
R\textsuperscript{2} in each test assumes a positive value, meaning that the pollutants taken into consideration in this case study can be explained by the regression models. 

\pagebreak
\subsection{Comparison with CAMS model}
The comparison between the models of my results and CAMS data shows how values are very far from them. MAE for PM2.5 values is around 5 while for ammonia around 10 ug/m\textsuperscript{3}. \\
Another difference between the two models is the R\textsuperscript{2} value which assumes a negative value in most cases. It proves the presence of a big difference between training and test values. 
\begin{table}[H]
\begin{tabular}{lrrrrr}
\toprule
  &  24/03-31/03 &  18/04-25/04 &  17/07-24/07 &  3/09-10/09 &  7/10-14/10 \\
\midrule
  MAE\_cams &        7.800 &        6.556 &        2.110 &       3.060 &       3.478 \\
  RMSE\_cams &        9.262 &        7.571 &        2.745 &       3.623 &       4.059 \\
   MSE\_cams &       86.279 &       57.593 &        7.548 &      13.266 &      16.591 \\
    R2\_cams &       -2.147 &       -7.492 &       -2.846 &      -0.258 &      -1.096 \\
\bottomrule
\end{tabular}
\caption{Random Forest prediction for PM2.5 at 1 km, including zones with mountains.}
\end{table}
\begin{table}[H]
\begin{tabular}{lrrrrr}
\toprule
  &  24/03-31/03 &  18/04-25/04 &  17/07-24/07 &  3/09-10/09 &  7/10-14/10 \\
\midrule
   MAE\_cams &       18.401 &       10.414 &       11.746 &      14.180 &      12.012 \\
  RMSE\_cams &       19.196 &       12.126 &       17.744 &      21.241 &      13.697 \\
   MSE\_cams &      368.858 &      147.652 &      323.418 &     456.145 &     188.650 \\
    R2\_cams &        0.136 &       -1.511 &       -0.158 &      -0.001 &      -0.060 \\
\bottomrule
\end{tabular}
\caption{Random Forest prediction for NH3 at 1 km, excluding zones with mountains.}
\end{table}
These results are in contrast with the FS results, where the correlation between the ARPA values (target variable) and the CAMS models is one of the strongest.
From this results, we hypothesize that both pm25\_st / nh3\_st and pm25\_cams / nh3\_cams variables describe the same pollution phenomenon having the same behaviour, but between them, there would be a bias.\\
Obviously, the model of Copernicus Atmosphere Monitoring Service provides different results than the one obtained by me, since is built very differently. CAMS is built through an ensemble median, averaged by eleven European air quality forecasting systems. Therefore, the median model should perform better than the individual model products \cite{riccio2007seeking}.
Anyway, through information collected on the Copernicus website, only a limited number of ground observations are detected in the Italian territory. As can be seen in figure \ref{fig:cams} in Italy there is only 2 observations in Lombardy for PM2.5 detection.\\
Nothing was found for Ammonia.
\begin{figure}[H]
    \centering
    \includegraphics[scale=0.2]{images/cams_obs.png}
    \caption{The following map includes a set of points layers representing the forecast performance of the station used by the ensemble median model of CAMS. In particular, the RMSE of them is shown \cite{camsobs}. 
}
    \label{fig:cams}
\end{figure}

So, by the validation results and the larger number of ground observations provided by ARPA in this data collection, we hypothesize that this model should perform better on this local scale.
\par
Based on these results, we can point out that Random Forest is more precise than the  Neural Network model and that the performance is higher for particulate matter than ammonia estimation.\\
Random Forest indeed is one of the most preferred also in the literature prediction models for its easy configuration.
\subsection{Importance of the FS}
The underlying tables provide model results if FS is before applied or not.

It can be observed that feature selection is meaningful for the pre-processing phase since aims at discarding eventually useless variables and taking only the ones really relevant to the target variable. \\
In the results obtained, we can detect that selection of variable with FS assume a key role for a sample of small size.\\ 
Indeed, in the tests with a resolution 10 Km model, there is a feasible difference between the data trained by variable from FS and the ones chosen randomly (Table \ref{fig:importance10km}). \\
This gap decreases with higher resolution (Table \ref{fig:importance1km}).
This is implied by the fact that the more the training size is sufficient, the less a feature selection procedure improves the accuracy \cite{chu2012does}.
\par
In conclusion, the difference between the results of this test and the ones run for each period show also the presence of overfitting in my models. Indeed, results obtained with temporal hold-out validation (in particular for  R\textsuperscript{2}) shows how the \acrshort{rf} model can't make accurate predictions about new data (which comes from a different period).  

\begin{table}[H]
\centering
\subfloat[10 km resolution.]
{\begin{tabular}{lrr}
\toprule
 &  With FS & Without FS \\
  \midrule
 MAE\_sensor &        2.123  &        2.061 \\
RMSE\_sensor &        2.645 &        2.567 \\
 MSE\_sensor &        6.997  &       6.592 \\
  R2\_sensor &        0.130  &         0.176\\
   Training time: & 2.402 s& 3.637 s\\
\end{tabular}
\label{fig:importance10km}}
\\
\subfloat[1 km resolution.]
{\begin{tabular}{lrr}
\toprule
& With FS &  Without FS\\
\midrule
 MAE\_sensor &        2.677 &        2.660 \\
RMSE\_sensor &        3.230  &        3.252\\
 MSE\_sensor &        10.434 &        10.575  \\
  R2\_sensor &        -0.268 &        -0.285 \\
   Training time & 17.298 s & 24.338 s\\
 \label{fig:importance1km}
\end{tabular}}
\caption{Random Forest prediction for PM2.5, including zones with mountains using or not FS.}
\end{table}

\chapter{Conclusion}
\label{chap:conclusion}
The aim of this chapter is to highlight the main outcomes that I have obtained in my work.
Based on the results achieved in the previous chapter (section \ref{sec:modelling2}), we can conclude that Machine Learning models performed differently for many factors such as the type of model, configuration, and resolution.
An important aspect shown in my work is the importance of the number of ground sensor observations used for training the models. In order to have enhanced estimation, reducing the lack of these is essential.\\
In this work I proposed a solution based on the use of KNN to increase the number of cells with interpolated values. An improved solution to increase the number of ground-truth observations is needed, without recurring to interpolation procedure.
In \acrshort{ml} models performance grows up as the model complexity, turning such systems into “black box” approaches and implying uncertainty in the way they work and come to decisions. 
This becomes a problematic challenge for machine learning systems to be used in critical domains, such as healthcare or the economic aspect.
The use of feature selection supports more efficient debugging for causalities and greater trust in the model.
The selection of features also attempts to clarify the decision of a model by determining the influence of each input variable. 
Scores of feature selection alone may not give a proper comprehension of the model’s reasoning, but it is partially helpful for interpreting the decisions taken.\\
Nevertheless, this case study was not brought to build a proper model for air pollution forecast; instead, it aims to detect how the selection of relevant features could affect the interpretability and the performance of the model. \\
The modelling task will be more properly made by an other collegue in the D-DUST project, by means of Machine Learning and geostatistic PM predictive models.\\
The present study confirmed the findings about the model performance which increases if a feature selection is applied, in particular when we have to deal with a limited sample of data \cite{vabalas2019machine}. 
In addition, what coming out from this is that the more the training sample size is limited, the more an accurate selection of the most weighted variables is needed to increase its performance.
We can say that the feature selection application is necessary but not a sufficient condition to have an increment in the model performance.
\begin{comment}
In this work, so it is highlighted the effect of how the training in \acrshort{ml} should benefit from an accurate selection of variables. 
\end{comment}
Instead of faultless building model with exact predictions, the results in this research pointed more towards having an interpretable model from the covariates chosen. 
One of the future outcomes from this is absolutely the importance of a model sufficiently explained.
Future research on AI should extend the explanations and interpretability of \acrshort{ml} models.
In high-risk applications, AI should not be blind. 
It's needed to dissect a model for proper comprehension and explanation.
For sure models like those could be helpful for the implementation of precise forecast models.  
For instance, this could be used as an important component for making an average through an ensemble technique for more complex and complete model (as it has been already done with the CAMS model).
\par
Finally, this work argued that by using feature selection it's possible to detect which are the main factors affecting the target variable and, possibly, to control them to reduce pollutants effects.\\
The score assumed by each different variable in the case study provides a measure of the influence in ammonia and fine particle formation, aspect which finds also confirmation in literature.
\begin{comment}
Looking forward, further attempts for reducing pollutant formation should be made by procedures actually used.
\end{comment}



%-------------------------------------------------------------------------
%	BIBLIOGRAPHY
%-------------------------------------------------------------------------

\addtocontents{toc}{\vspace{2em}} % Add a gap in the Contents, for aesthetics
\bibliography{Thesis_bibliography} % The references information are stored in the file named "Thesis_bibliography.bib"

%-------------------------------------------------------------------------
%	APPENDICES
%-------------------------------------------------------------------------

\cleardoublepage
\addtocontents{toc}{\vspace{2em}} % Add a gap in the Contents, for aesthetics
\appendix
\chapter{Appendix}
\label{chap:appendix}
In this appendix are collected all results of
\begin{itemize}
    \item feature selection evaluated by fs\_results.ipynb;
    \item model prediction errors computed with:
    \begin{itemize}
        \item Keras\_prediction\_model.ipynb;
        \item RandomForest\_prediction\_model.ipynb;
    \end{itemize}  
\end{itemize}
\section{Feature Selection results}
\subsection{Borda Count results}
\begin{figure}[H]
\centering
\subfloat[10 Km resolution with mountains]{\includegraphics[scale =0.40]{images/tests/0_1_mountainspm25_st.png}}\\
\subfloat[10 Km resolution with mountains]{\includegraphics[scale =0.40]{images/tests/0_1_nomountainspm25_st.png}}
\caption{FS results obtained with fine particulate ('pm25\_st') as target variable and 10 km resolution. The results are averaged over the 5 periods. }
\end{figure}
\begin{figure}[H]
\centering
\subfloat[1 Km resolution with mountains]{\includegraphics[scale =0.42]{images/tests/0_01_mountainspm25_st.png}}\\
\subfloat[1 Km resolution with mountains]{\includegraphics[scale =0.42]{images/tests/0_01_nomountainspm25_st.png}}
\caption{FS results obtained with fine particulate ('pm25\_st') as target variable and 1 km resolution. The results are averaged over the 5 periods.}
\end{figure}
\begin{figure}[H]
\centering
\subfloat[10 Km resolution with mountains]{\includegraphics[scale =0.42]{images/tests/0_1_mountainsnh3_st.png}}\\
\subfloat[10 Km resolution with mountains]{\includegraphics[scale =0.42]{images/tests/0_1_nomountainsnh3_st.png}}
\caption{FS results obtained with ammonia ('nh3\_st') as target variable and 10 km resolution. The results are averaged over the 5 periods.}
\end{figure}
\begin{figure}[H]
\centering
\subfloat[1 Km resolution with mountains]{\includegraphics[scale =0.42]{images/tests/0_01_mountainsnh3_st.png}}\\
\subfloat[1 Km resolution with mountains]{\includegraphics[scale =0.42]{images/tests/0_01_nomountainsnh3_st.png}}
\caption{FS results obtained with ammonia ('nh3\_st') as target variable and 1 km resolution. The results are averaged over the 5 periods.}
\end{figure}
\subsection{Pearson correlation index results}
\begin{figure}[H]
    \centering
    \includegraphics[scale=0.38]{images/tests/0_1_mountainspm25_st_pearson.png}
    \caption{Pearson correlation index results with respect to fine particulate ('pm25\_st') as target variable, with 10km resolution including mountains in each period. }
\end{figure}
\begin{figure}[H]
    \centering
    \includegraphics[scale=0.38]{images/tests/0_1_nomountainspm25_st_pearson.png}
    \caption{Pearson correlation index results with respect to fine particulate ('pm25\_st') as target variable, with 10km resolution excluding mountains in each period.}
    
\end{figure}
\begin{figure}[H]
    \centering
    \includegraphics[scale=0.38]{images/tests/0_01_mountainspm25_st_pearson.png}
    \caption{Pearson correlation index results with respect to fine particulate ('pm25\_st') as target variable, with 1km resolution including mountains in each period.}
    
\end{figure}
\begin{figure}[H]
    \centering
    \includegraphics[scale=0.38]{images/tests/0_01_nomountainspm25_st_pearson.png}
    \caption{Pearson correlation index results with respect to fine particulate ('pm25\_st') as target variable, with 1km resolution excluding mountains in each period.}
    
\end{figure}


\begin{figure}[H]
    \centering
    \includegraphics[scale=0.38]{images/tests/0_1_mountainsnh3_st_pearson.png}
    \caption{Pearson correlation index results with respect to ammonia ('nh3\_st') as target variable, with 10km resolution including mountains in each period.}
    
\end{figure}
\begin{figure}[H]
    \centering
    \includegraphics[scale=0.38]{images/tests/0_1_nomountainsnh3_st_pearson.png}
    \caption{Pearson correlation index results with respect to ammonia ('nh3\_st') as target variable, with 10km resolution excluding mountains in each period.}
    
\end{figure}
\begin{figure}[H]
    \centering
    \includegraphics[scale=0.38]{images/tests/0_01_mountainsnh3_st_pearson.png}
    \caption{Pearson correlation index results with respect to ammonia ('nh3\_st') as target variable, with 1km resolution including mountains in each period.}
    
\end{figure}
\begin{figure}[H]
    \centering
    \includegraphics[scale=0.38]{images/tests/0_01_nomountainsnh3_st_pearson.png}
    \caption{Pearson correlation index results with respect to ammonia ('nh3\_st') as target variable, with 1km resolution excluding mountains in each period.}
    
\end{figure}

\section{ML models results}
\subsection{Random Forest}
\begin{table}[H]
\begin{tabular}{lrrrrr}
\toprule
 &  24/03-31/03 &  18/04-25/04 &  17/07-24/07 &  3/09-10/09 &  7/10-14/10 \\
\midrule
 MAE\_sensor &        1.239 &        0.900 &        0.566 &       0.987 &       0.751 \\
RMSE\_sensor &        1.768 &        1.133 &        0.745 &       1.330 &       1.013 \\
 MSE\_sensor &        3.215 &        1.295 &        0.571 &       1.802 &       1.067 \\
  R2\_sensor &        0.883 &        0.812 &        0.722 &       0.834 &       0.858 \\
   MAE\_cams &        7.800 &        6.556 &        2.110 &       3.060 &       3.478 \\
  RMSE\_cams &        9.262 &        7.571 &        2.745 &       3.623 &       4.059 \\
   MSE\_cams &       86.279 &       57.593 &        7.548 &      13.266 &      16.591 \\
    R2\_cams &       -2.147 &       -7.492 &       -2.846 &      -0.258 &      -1.096 \\
\bottomrule
\end{tabular}
\caption{Random Forest prediction for PM2.5 with 10 km resolution, including zones with mountains.}
\end{table}
\begin{table}[H]
\begin{tabular}{lrrrrr}
\toprule
 &  24/03-31/03 &  18/04-25/04 &  17/07-24/07 &  3/09-10/09 &  7/10-14/10 \\
\midrule
 MAE\_sensor &        1.448 &        1.093 &        0.646 &       0.898 &       0.841 \\
RMSE\_sensor &        1.940 &        1.362 &        0.868 &       1.165 &       1.080 \\
 MSE\_sensor &        3.772 &        1.874 &        0.776 &       1.376 &       1.197 \\
  R2\_sensor &        0.872 &        0.766 &        0.683 &       0.870 &       0.793 \\
   MAE\_cams &        8.230 &        7.914 &        1.386 &       3.581 &       3.744 \\
  RMSE\_cams &        9.663 &        8.714 &        1.672 &       4.105 &       4.336 \\
   MSE\_cams &       95.109 &       76.441 &        2.867 &      16.923 &      18.964 \\
    R2\_cams &       -2.312 &       -8.546 &       -0.185 &      -0.632 &      -2.373 \\
\bottomrule
\bottomrule
\end{tabular}
\caption{Random Forest prediction for PM2.5 with 10 km resolution, excluding zones with mountains.}
\end{table}
\begin{table}[H]
\begin{tabular}{lrrrrr}
\toprule
 &  24/03-31/03 &  18/04-25/04 &  17/07-24/07 &  3/09-10/09 &  7/10-14/10 \\
\midrule
 MAE\_sensor &        0.251 &        0.186 &        0.147 &       0.172 &       0.132 \\
RMSE\_sensor &        0.503 &        0.361 &        0.255 &       0.285 &       0.240 \\
 MSE\_sensor &        0.283 &        0.138 &        0.066 &       0.084 &       0.059 \\
  R2\_sensor &        0.992 &        0.985 &        0.981 &       0.994 &       0.993 \\
   MAE\_cams &        8.966 &        7.566 &        1.936 &       3.574 &       3.974 \\
  RMSE\_cams &       10.310 &        8.472 &        2.487 &       4.099 &       4.580 \\
   MSE\_cams &      106.414 &       71.818 &        6.213 &      16.810 &      20.988 \\
    R2\_cams &       -1.821 &       -7.099 &       -0.727 &      -0.180 &      -1.578 \\
\bottomrule
\end{tabular}
\caption{Random Forest prediction for PM2.5 with 1 km resolution, including zones with mountains.}
\end{table}
\begin{table}[H]
\begin{tabular}{lrrrrr}
\toprule
 &  24/03-31/03 &  18/04-25/04 &  17/07-24/07 &  3/09-10/09 &  7/10-14/10 \\
\midrule
 MAE\_sensor &        0.347 &        0.234 &        0.156 &       0.199 &       0.146 \\
RMSE\_sensor &        0.624 &        0.392 &        0.261 &       0.327 &       0.232 \\
 MSE\_sensor &        0.411 &        0.158 &        0.069 &       0.113 &       0.054 \\
  R2\_sensor &        0.990 &        0.985 &        0.982 &       0.993 &       0.991 \\
   MAE\_cams &        9.153 &        8.580 &        1.617 &       3.830 &       4.192 \\
  RMSE\_cams &       10.564 &        9.294 &        1.940 &       4.382 &       4.760 \\
   MSE\_cams &      111.721 &       86.530 &        3.766 &      19.284 &      22.676 \\
    R2\_cams &       -1.682 &       -7.131 &        0.004 &      -0.187 &      -2.674 \\
\bottomrule
\end{tabular}
\caption{Random Forest prediction for PM2.5 with 1 km resolution, excluding zones with mountains.}
\end{table}
\begin{table}[H]
\begin{tabular}{lrrrrr}
\toprule
  &  24/03-31/03 &  18/04-25/04 &  17/07-24/07 &  3/09-10/09 &  7/10-14/10 \\
\midrule
 MAE\_sensor &        4.601 &        2.037 &        2.401 &       3.634 &       1.932 \\
RMSE\_sensor &        5.827 &        2.809 &        3.325 &       5.074 &       2.883 \\
 MSE\_sensor &       35.506 &        8.008 &       11.853 &      26.633 &       9.053 \\
  R2\_sensor &        0.853 &        0.815 &        0.922 &       0.854 &       0.801 \\
   MAE\_cams &       14.956 &       10.748 &        8.166 &      10.326 &       9.163 \\
  RMSE\_cams &       16.458 &       11.633 &       12.027 &      13.395 &      10.105 \\
   MSE\_cams &      272.549 &      135.617 &      149.976 &     194.430 &     102.945 \\
    R2\_cams &       -0.462 &       -2.048 &        0.013 &       0.139 &      -1.066 \\
\bottomrule
\end{tabular}
\caption{Random Forest prediction for NH3 with 10 km resolution, including zones with mountains.}
\end{table}
\begin{table}[H]
\begin{tabular}{lrrrrr}
\toprule
 &  24/03-31/03 &  18/04-25/04 &  17/07-24/07 &  3/09-10/09 &  7/10-14/10 \\
\midrule
 MAE\_sensor &        4.039 &        1.891 &        3.524 &       3.780 &       1.812 \\
RMSE\_sensor &        5.365 &        2.496 &        4.398 &       5.358 &       2.601 \\
 MSE\_sensor &       29.553 &        6.521 &       22.504 &      30.148 &       7.815 \\
  R2\_sensor &        0.895 &        0.829 &        0.814 &       0.848 &       0.911 \\
   MAE\_cams &       15.383 &       10.758 &        8.942 &      10.304 &       9.497 \\
  RMSE\_cams &       16.887 &       11.705 &       12.639 &      13.799 &      10.497 \\
   MSE\_cams &      288.521 &      138.082 &      178.690 &     211.569 &     110.820 \\
    R2\_cams &       -0.022 &       -2.954 &       -0.236 &       0.062 &      -0.547 \\
\bottomrule
\end{tabular}
\caption{Random Forest prediction for NH3 with 10 km resolution, excluding zones with mountains.}
\end{table}
\begin{table}[H]
\begin{tabular}{lrrrrr}
\toprule
 &  24/03-31/03 &  18/04-25/04 &  17/07-24/07 &  3/09-10/09 &  7/10-14/10 \\
\midrule
 MAE\_sensor &        0.450 &        0.180 &        0.505 &       0.669 &       0.345 \\
RMSE\_sensor &        0.954 &        0.327 &        1.120 &       1.803 &       0.785 \\
 MSE\_sensor &        1.065 &        0.147 &        1.861 &       4.244 &       0.732 \\
  R2\_sensor &        0.998 &        0.998 &        0.994 &       0.990 &       0.995 \\
   MAE\_cams &       17.694 &       10.536 &       11.049 &      13.329 &      11.480 \\
  RMSE\_cams &       18.513 &       12.046 &       16.887 &      19.503 &      12.945 \\
   MSE\_cams &      343.112 &      145.433 &      287.761 &     382.879 &     169.030 \\
    R2\_cams &        0.227 &       -1.134 &       -0.044 &       0.124 &      -0.030 \\
\bottomrule
\end{tabular}
\caption{Random Forest prediction for NH3 with 1 km resolution, including zones with mountains.}
\end{table}
\begin{table}[H]
\begin{tabular}{lrrrrr}
\toprule
 &  24/03-31/03 &  18/04-25/04 &  17/07-24/07 &  3/09-10/09 &  7/10-14/10 \\
\midrule
 MAE\_sensor &        0.483 &        0.280 &        0.526 &       0.939 &       0.386 \\
RMSE\_sensor &        0.993 &        0.558 &        1.184 &       2.316 &       0.860 \\
 MSE\_sensor &        1.199 &        0.430 &        1.556 &       6.147 &       1.121 \\
  R2\_sensor &        0.997 &        0.992 &        0.994 &       0.987 &       0.990 \\
   MAE\_cams &       18.401 &       10.414 &       11.746 &      14.180 &      12.012 \\
  RMSE\_cams &       19.196 &       12.126 &       17.744 &      21.241 &      13.697 \\
   MSE\_cams &      368.858 &      147.652 &      323.418 &     456.145 &     188.650 \\
    R2\_cams &        0.136 &       -1.511 &       -0.158 &      -0.001 &      -0.060 \\
\bottomrule
\end{tabular}
\caption{Random Forest prediction for NH3 with 1 km resolution, excluding zones with mountains.}
\end{table}
\subsection{Neural Network by Keras}
\begin{table}[H]
\begin{tabular}{lrrrrr}
\toprule
 &  24/03-31/03 &  18/04-25/04 &  17/07-24/07 &  3/09-10/09 &  7/10-14/10 \\
\midrule
 MAE\_sensor &        2.127 &        1.581 &        0.847 &       1.516 &       1.268 \\
RMSE\_sensor &        2.643 &        2.008 &        1.046 &       1.912 &       1.591 \\
 MSE\_sensor &        7.097 &        4.064 &        1.108 &       3.743 &       2.578 \\
  R2\_sensor &        0.750 &        0.358 &        0.408 &       0.649 &       0.672 \\
   MAE\_cams &        7.800 &        6.554 &        2.113 &       3.060 &       3.478 \\
  RMSE\_cams &        9.262 &        7.571 &        2.720 &       3.598 &       4.067 \\
   MSE\_cams &       86.279 &       57.562 &        7.565 &      13.266 &      16.591 \\
    R2\_cams &       -2.147 &       -8.102 &       -3.386 &      -0.239 &      -1.158 \\
\bottomrule
\end{tabular}
\caption{Neural Network prediction for PM2.5 with 10 km resolution, including zones with mountains.}
\end{table}
\begin{table}[H]
\begin{tabular}{lrrrrr}
\toprule
 &  24/03-31/03 &  18/04-25/04 &  17/07-24/07 &  3/09-10/09 &  7/10-14/10 \\
\midrule
 MAE\_sensor &        1.864 &        1.605 &        0.871 &       1.308 &       1.052 \\
RMSE\_sensor &        2.257 &        1.955 &        1.100 &       1.612 &       1.327 \\
 MSE\_sensor &        5.188 &        3.894 &        1.279 &       2.696 &       1.890 \\
  R2\_sensor &        0.826 &        0.538 &        0.457 &       0.731 &       0.653 \\
   MAE\_cams &        8.243 &        7.907 &        1.383 &       3.581 &       3.744 \\
  RMSE\_cams &        9.717 &        8.713 &        1.677 &       4.086 &       4.342 \\
   MSE\_cams &       95.313 &       76.364 &        2.859 &      16.923 &      18.964 \\
    R2\_cams &       -2.281 &       -8.213 &       -0.193 &      -0.561 &      -2.351 \\
\bottomrule
\end{tabular}
\caption{Neural Network prediction for PM2.5 with 10 km resolution, excluding zones with mountains.}
\end{table}
\begin{table}[H]
\begin{tabular}{lrrrrr}
\toprule
 &  24/03-31/03 &  18/04-25/04 &  17/07-24/07 &  3/09-10/09 &  7/10-14/10 \\
\midrule
 MAE\_sensor &        1.546 &        0.970 &        0.721 &       1.170 &       1.040 \\
RMSE\_sensor &        1.997 &        1.340 &        0.988 &       1.528 &       1.325 \\
 MSE\_sensor &        4.079 &        1.831 &        0.991 &       2.339 &       1.776 \\
  R2\_sensor &        0.891 &        0.798 &        0.727 &       0.839 &       0.779 \\
   MAE\_cams &        8.966 &        7.567 &        1.936 &       3.574 &       3.974 \\
  RMSE\_cams &       10.314 &        8.473 &        2.490 &       4.096 &       4.579 \\
   MSE\_cams &      106.423 &       71.820 &        6.214 &      16.811 &      20.987 \\
    R2\_cams &       -1.840 &       -6.979 &       -0.735 &      -0.157 &      -1.585 \\
\bottomrule
\end{tabular}
\caption{Neural Network prediction for PM2.5 with 1 km resolution, including zones with mountains.}
\end{table}
\begin{table}[H]
\begin{tabular}{lrrrrr}
\toprule
 &  24/03-31/03 &  18/04-25/04 &  17/07-24/07 &  3/09-10/09 &  7/10-14/10 \\
\midrule
 MAE\_sensor &        1.508 &        0.875 &        0.739 &       0.793 &       0.593 \\
RMSE\_sensor &        1.995 &        1.174 &        0.972 &       1.002 &       0.836 \\
 MSE\_sensor &        4.033 &        1.396 &        0.967 &       1.050 &       0.704 \\
  R2\_sensor &        0.904 &        0.868 &        0.737 &       0.937 &       0.888 \\
   MAE\_cams &        9.153 &        8.581 &        1.617 &       3.830 &       4.192 \\
  RMSE\_cams &       10.565 &        9.298 &        1.939 &       4.386 &       4.761 \\
   MSE\_cams &      111.720 &       86.539 &        3.766 &      19.282 &      22.673 \\
    R2\_cams &       -1.627 &       -7.082 &       -0.006 &      -0.181 &      -2.616 \\
\bottomrule
\end{tabular}
\caption{Neural Network prediction for PM2.5 with 1 km resolution, excluding zones with mountains.}
\end{table}
\begin{table}[H]
\begin{tabular}{lrrrrr}
\toprule
 &  24/03-31/03 &  18/04-25/04 &  17/07-24/07 &  3/09-10/09 &  7/10-14/10 \\
\midrule
 MAE\_sensor &        5.441 &        2.433 &        4.971 &       5.729 &       3.219 \\
RMSE\_sensor &        6.896 &        3.283 &        7.496 &       7.618 &       4.613 \\
 MSE\_sensor &       49.263 &       11.191 &       60.601 &      70.032 &      21.714 \\
  R2\_sensor &        0.785 &        0.722 &        0.618 &       0.726 &       0.722 \\
   MAE\_cams &        6.660 &        4.050 &        7.537 &       6.583 &       2.601 \\
  RMSE\_cams &        8.289 &        5.473 &       10.610 &       9.773 &       3.662 \\
   MSE\_cams &       69.718 &       31.198 &      115.937 &     102.716 &      13.809 \\
    R2\_cams &        0.680 &        0.217 &        0.212 &       0.546 &       0.822 \\
\bottomrule
\end{tabular}
\caption{Neural Network prediction for NH3 with 10 km resolution, including zones with mountains.}
\end{table}
\begin{table}[H]
\begin{tabular}{lrrrrr}
\toprule
 &  24/03-31/03 &  18/04-25/04 &  17/07-24/07 &  3/09-10/09 &  7/10-14/10 \\
\midrule
 MAE\_sensor &        4.625 &        2.297 &        4.749 &       5.573 &       3.139 \\
RMSE\_sensor &        5.405 &        2.988 &        5.801 &       7.571 &       4.557 \\
 MSE\_sensor &       33.849 &        9.630 &       37.672 &      58.766 &      25.848 \\
  R2\_sensor &        0.867 &        0.710 &        0.702 &       0.704 &       0.675 \\
   MAE\_cams &        7.283 &        4.559 &        8.412 &       7.581 &       3.033 \\
  RMSE\_cams &        8.867 &        6.016 &       11.236 &      10.874 &       4.072 \\
   MSE\_cams &       78.979 &       36.499 &      135.704 &     123.934 &      16.818 \\
    R2\_cams &        0.710 &       -0.106 &       -0.034 &       0.425 &       0.591 \\
\bottomrule
\end{tabular}
\caption{Neural Network prediction for NH3 with 10 km resolution, excluding zones with mountains.}
\end{table}
\begin{table}[H]
\begin{tabular}{lrrrrr}
\toprule
 &  24/03-31/03 &  18/04-25/04 &  17/07-24/07 &  3/09-10/09 &  7/10-14/10 \\
\midrule
 MAE\_sensor &        3.064 &        1.920 &        4.194 &       2.612 &       2.140 \\
RMSE\_sensor &        3.953 &        3.010 &        5.592 &       3.497 &       2.518 \\
 MSE\_sensor &       17.004 &        9.733 &       31.882 &      12.736 &       8.747 \\
  R2\_sensor &        0.962 &        0.861 &        0.879 &       0.968 &       0.953 \\
   MAE\_cams &        8.197 &        4.031 &       10.073 &       9.247 &       4.327 \\
  RMSE\_cams &       10.860 &        6.169 &       14.998 &      15.436 &       7.145 \\
   MSE\_cams &      118.847 &       38.357 &      225.879 &     243.268 &      51.953 \\
    R2\_cams &        0.736 &        0.456 &        0.180 &       0.445 &       0.695 \\
\bottomrule
\end{tabular}
\caption{Neural Network prediction for NH3 with 1 km resolution, including zones with mountains.}
\end{table}


\begin{table}[H]
\begin{tabular}{lrrrrr}
\toprule
 &  24/03-31/03 &  18/04-25/04 &  17/07-24/07 &  3/09-10/09 &  7/10-14/10 \\
\midrule
 MAE\_sensor &        1.904 &        2.287 &        3.126 &       2.743 &       3.146 \\
RMSE\_sensor &        2.570 &        3.341 &        4.205 &       3.841 &       4.465 \\
 MSE\_sensor &        6.867 &       11.579 &       18.068 &      16.405 &      21.579 \\
  R2\_sensor &        0.984 &        0.800 &        0.927 &       0.965 &       0.852 \\
   MAE\_cams &        9.253 &        4.560 &       11.334 &      11.259 &       5.461 \\
  RMSE\_cams &       11.641 &        6.600 &       15.831 &      17.476 &       8.145 \\
   MSE\_cams &      135.675 &       43.816 &      257.910 &     306.085 &      67.695 \\
    R2\_cams &        0.692 &        0.254 &        0.058 &       0.342 &       0.631 \\
\bottomrule
\end{tabular}
\caption{Neural Network prediction for NH3 with 1 km resolution, excluding zones with mountains.}
\end{table}












% LIST OF FIGURES
\listoffigures

% LIST OF TABLES
\listoftables

% LIST OF SYMBOLS
% Write out the List of Symbols in this page
\chapter*{List of Symbols} % You have to include a chapter for your list of symbols (
\begin{table}[H]
    \centering
    \begin{tabular}{lll}
        \textbf{Variable} & \textbf{Description} & \textbf{SI unit} \\\hline\\[-9px]
        $\bm{u}$ & solid displacement & m \\[2px]
        $\bm{u}_f$ & fluid displacement & m \\[2px]
    \end{tabular}
\end{table}

% ACKNOWLEDGEMENTS
\chapter*{Acknowledgements}
Here you might want to acknowledge someone.

\cleardoublepage

\end{document}