L'arrivo dei big data da fonti diverse ha avuto un grande impatto anche negli studi ambientali, tra cui l'analisi dell'inquinamento dell'aria.\\
Allo stesso tempo tecniche come il \gls{ml} diventano gli approcci standard per fare previsioni.
Ma dataset numerosi e con troppe variabili possono rendere un modello d'intelligenza artificiale più complesso e difficile da essere interpretato e spiegato.\par
Un difetto dei modelli di Machine Learning è spesso la sua natura dell'essere una scatola chiusa (black box). La complessità della sua struttura impedisce all'essere umano di comprenderne e spiegarne appieno il suo funzionamento.\\
Dovrebbe essere importante determinare le ragioni delle decisioni di un modello, specialmente in scenari rischiosi come l'assistenza medica e attività di policy-making.
Un modo per interpretare i modelli è quello di confrontare l'importanza di ogni variabile: può essere utile dare una gerarchia ad ognuna di esse per trovare quelle più influenti. \par
Questa tesi ha lo scopo di contribuire alla fase di preprocessing di un modello ML, considerandola essenziale prima della sua fase di training.\\
Lo scopo di questo lavoro è di fornire un approccio per trovare le migliori variabili predittive tramite l'uso della feature selection. \\
Il suo vantaggio è quello di aiutare a ridurre la ridondanza ed aumentare l'interpretabilità in un modello.
Questa tesi presenta un insieme di strumenti per processare i dati e selezionare le variabili più influenti tramite metodi filter, wrapper e embedded di feature selection.
In particolare questa tesi è stata applicata in un caso studio del progetto D-DUST che ha il fine di analizzare l'impatto che hanno le attività agricole intensive in Lombardia (Nord Italia) sulla qualità dell'aria.\\ 
Alcuni modelli ML per la predizioni della concentrazione degli inquinanti sono stati implementati per comparare l'accuratezza e l'interpretabilità dei risultati.\\
I risultati ottenuti con varie configurazioni di Feature Selection sono stati analizzati e successivamente confrontati con quelli trovati nella letteratura scientifica.
\\
