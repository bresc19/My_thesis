L'arrivo dei big data da fonti diverse ha avuto un grande impatto sull'analisi dell'inquinamento.\\
Allo stesso tempo tecniche come il \gls{ml} diventano quelle più utilizzate per fare previsioni.
Ma dataset numerosi e con troppe variabili possono rendere un modello d'intelligenza artificiale più complesso e difficile da interpretare.\par
Un difetto dei modelli di Machine Learning è spesso la sua natura dell'essere una scatola chiusa (black box). La complessità della sua struttura impedisce all'essere umano di comprenderne e spiegarne appieno il suo funzionamento.\\
Dovrebbe essere importante determinare le ragioni delle decisioni di un modello, specialmente in scenari rischiosi come le diagnosi mediche.
Un modo per interpretare i modelli è quello di confrontare l'importanza di ogni variabile: può essere utile dare una gerarchia ad ognuna di esse per distinguere quelle più influenti rispetto a quelle che potrebbero "confondere" le decisioni prese dal modello. \par
Questa tesi vuole contribuire alla fase di preprocessing di un modello di Machine Learning, considerandola essenziale prima della sua fase di training.\\
Lo scopo di questo lavoro è di fornire un approccio per trovare le migliori variabili predittive tramite l'uso della feature selection. \\
Il suo vantaggio è quello di aiutare a ridurre la ridondanza ed aumentare l'interpretabilità in un modello.
Questa tesi presenta un insieme di strumenti per processare i dati e selezionare le variabili più influenti tramite metodi filter, wrapper e embedded di feature selection.
In particolare questa tesi è stata applicata in un caso studio del progetto D-DUST che ha il fine di analizzare l'impatto che ha l'agricoltura intensiva in Lombardia.\\ 
Alcuni modelli di Machine Learning sono stati implementati per comparare l'accuratezza e l'interpretabilità dei risultati.\\
I risultati ottenuti con varie configurazioni e risoluzioni sono stati analizzati e successivamente confrontati con gli analoghi trovati nella letterature scientifica.
\\
