The arrival of big data from heterogeneous sources has had an enormous impact on pollution analysis.\\
At the same time, techniques such as Machine Learning models become more and more the main approaches to make predictions. But with data sets with large samples and covariates, AI models could be more complex to interpret and explain.\par
One machine learning issue is the fact that models are usually a kind of “black box”: systems that evaluate output are not fully understandable by humans. \\
It should be important to interpret the prediction, especially for critical decisions such as healthcare diagnostics.
A way to interpret models for better reliability is to compare feature importance: giving a hierarchy of features could be very useful to distinguish the most influential variables from the ones which could confound model decisions.\par
This thesis wants to contribute to the preprocessing phase that is performed for \gls{ml} models, considering it essential before model training.\\
The challenge of this work is to propose an approach based on the concept of finding the best potential predictive variables with feature selection.\\
The advantage of using it is that the selected feature helps reduce redundancy and increase the interpretability of subsequent predictive models.
In detail, this thesis presents a set of tools to preprocess data and select variables with filter, wrapper, and embedded feature selection methods.\par
In this thesis, this approach has been applied in a case study of D-DUST project, which aims to analyze the impact of intense agriculture on air pollution in Lombardy.\\
Some ML models are also developed to compare the contribution to the accuracy and interpretability of the results.\\
The experimental results obtained by tests with different configurations and resolutions are analyzed and compared using analogous scientific literature.
\\
\\