The arrival of big data from heterogeneous sources has had an enormous impact in pollution analysis.
At the same time, techniques such as Machine Learning models become more and more the main approaches to make prediction and to detect the main factors affecting a given target variable.
In particular, with data sets with large sample and covariates, AI models are more complex to interpret and explain.
One machine learning issue is the fact that usually models are a kind of “black box”: systems that evaluate output not fully understandable by humans. 
It should be important to interpret the prediction, especially for critical decisions such as health state diagnostics.
A way to interpret models for better reliability is to compare feature importance: giving a hierarchy of features could be very useful to distinguish the most influential variables from the ones which could confound model decisions.
The challenge of this thesis is to propose an approach based on the concept of finding the best potential predictive variables based on the weighted score.
The advantage of using it is that the selected feature reduces redundancy and increases interpretability. 
Subsequent predictive models should have been improved in terms of accuracy and confidence in the results.
In detail, this thesis presents a set of tools to preprocess data and select variables with filter, wrapper, and embedded methods.
Some ML models are also developed with the aim of comparing the contribution to the accuracy of the results.
The experimental results obtained by tests with different configurations and resolutions are analyzed and compared using an analogous scientific literature.
This thesis wants to contribute to the preprocessing phase which is performed for Machine Learning models, considering it essential before regression model training.

Todo:
-conclusion 
-abstract
-knn values in thesis
-knn
-th value
-number of covariates
