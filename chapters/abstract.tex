The arrival of big data from heterogeneous sources has had an enormous impact also on environmental studies, including air pollution analysis.\\
At the same time, techniques such as \gls{ml} have increasingly become standard approaches to making predictions. But with data sets with large samples and covariates, AI models could be more complex to be interpreted and explained.\par
One machine learning issue is that models are usually “black box”: systems of which output are not fully understandable by humans. \\
It should be important to interpret the prediction, especially for critical decisions such as healthcare and policy making.
A way to interpret models for better reliability is to compare feature importance: giving a hierarchy of features could be very useful to detect the most influential variables.\par
This thesis work aims at contributing to the preprocessing phase that is performed for ML models, considering it essential before model training.\\
The challenge of this work is to propose an approach based on the concept of finding the best potential predictive variables with feature selection.\\
The advantage of using it is that the selected features help reduce redundancy and increase the interpretability of subsequent predictive models.
In detail, this thesis presents a set of tools to preprocess data and select variables with filter, wrapper, and embedded feature selection methods.\par
This approach has been applied in a case study of the D-DUST project, which aims to analyze the impact of intensive farming activities on air quality in the Lombardy region (Northern Italy).\\
Some ML model predictions of air pollutants concentrations are also developed to compare the contribution to the accuracy and interpretability of the results.\\
The experimental results obtained with different feature selection configurations are analyzed and compared with reference results available in the scientific literature.
\\
\\